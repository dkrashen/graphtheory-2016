\section{Two examples}{}{}
\nobreak
Up to now we have been concerned with extracting information about how
a function changes from the function itself. Given knowledge about an
object's position, for example, we want to know the object's
speed. Given information about the height of a curve we want to know
its slope. We now consider problems that are, whether obviously or
not, the reverse of such problems.

\begin{example}
An object moves in a straight line so that
its speed at time $t$ is given by $v(t)=3t$ in, say, cm/sec. If the
object is at position $10$ on the straight line when $t=0$, where is
the object at any time $t$? 

There are two reasonable ways to approach this problem. If $s(t)$ is
the position of the object at time $t$, we know that
$s'(t)=v(t)$. Because of our knowledge of derivatives, we know
therefore that $\ds s(t)=3t^2/2+k$, and because $s(0)=10$ we easily
discover that $k=10$, so $\ds s(t)=3t^2/2+10$. For example, at $t=1$ the
object is at position $3/2+10=11.5$.
This is certainly the easiest way to deal with this problem. Not all
similar problems are so easy, as we will see; the second approach to
the problem is more difficult but also more general.

We start by considering how we might approximate a solution. We know
that at $t=0$ the object is at position 10. How might we approximate
its position at, say, $t=1$? We know that the speed of the object at
time $t=0$ is $0$; if its speed were constant then in the first second
the object would not move and its position would still be 10 when
$t=1$. In fact, the object will not be too far from 10 at $t=1$, but
certainly we can do better. Let's look at the times $0.1$, $0.2$,
$0.3$, \dots, $1.0$, and try approximating the location of the object
at each, by supposing that during each tenth of a second the object is
going at a constant speed. Since the object initially has speed 0, we
again suppose it maintains this speed, but only for a tenth of second;
during that time the object would not move. During the tenth of a
second from $t=0.1$ to $t=0.2$, we suppose that the object is
traveling at $0.3$ cm/sec, namely, its actual speed at $t=0.1$. In
this case the object would travel $(0.3)(0.1)=0.03$ centimeters: $0.3$
cm/sec times $0.1$ seconds. Similarly, between $t=0.2$ and $t=0.3$ the
object would travel $(0.6)(0.1)=0.06$ centimeters.  Continuing, we get
as an approximation that the object travels
$$
  (0.0)(0.1)+(0.3)(0.1)+(0.6)(0.1)+\cdots+(2.7)(0.1)=1.35
$$ 
centimeters, ending up at position 11.35. This is a better
approximation than 10, certainly, but is still just an
approximation. (We know in fact that the object ends up at position
$11.5$, because we've already done the problem using the first
approach.) Presumably, we will get a better approximation if we divide
the time into one hundred intervals of a hundredth of a second each,
and repeat the process:
$$
  (0.0)(0.01)+(0.03)(0.01)+(0.06)(0.01)+\cdots+(2.97)(0.01)=1.485.
$$
We thus approximate the position as $11.485$. Since we know the exact
answer, we can see that this is much closer, but if we did not already
know the answer, we wouldn't really know how close.

We can keep this up, but we'll never really know the exact answer if
we simply compute more and more examples. Let's instead look at a
``typical'' approximation. Suppose we divide the time into $n$ equal
intervals, and imagine that on each of these the object travels at a
constant speed. Over the first time interval we approximate the
distance traveled as $(0.0)(1/n)=0$, as before. During the second time
interval, from $t=1/n$ to $t=2/n$, the object travels approximately
$\ds 3(1/n)(1/n)=3/n^2$ centimeters. During time interval number $i$, the
object travels approximately $\ds (3(i-1)/n)(1/n)=3(i-1)/n^2$
centimeters, that is, its speed at time $(i-1)/n$, $3(i-1)/n$, times
the length of time interval number $i$, $1/n$.
Adding these up as before, we approximate the distance traveled as
$$
  (0){1\over n}+3{1\over n^2}+3(2){1\over n^2}+
  3(3){1\over n^2}+\cdots+3(n-1){1\over n^2}
$$
centimeters. What can we say about this? At first it looks rather less
useful than the concrete calculations we've already done. But in fact
a bit of algebra reveals it to be much more useful. We can factor out
a 3 and $\ds 1/n^2$ to get
$$
  {3\over n^2}(0+1+2+3+\cdots+(n-1)),
$$
that is, $\ds 3/n^2$ times the sum of the first $n-1$ positive
integers. Now we make use of a fact you may have run across before:
$$
  1+2+3+\cdots+k={k(k+1)\over2}.
$$
In our case we're interested in $k=n-1$, so
$$
  1+2+3+\cdots+(n-1)={(n-1)(n)\over2}={n^2-n\over2}.
$$
This simplifies the approximate distance traveled to 
$$
  {3\over n^2}{n^2-n\over2}={3\over2}{n^2-n\over n^2}=
  {3\over2}\left({n^2\over n^2}-{n\over n^2}\right)=
  {3\over2}\left(1-{1\over n}\right).
$$
Now this is quite easy to understand: as $n$ gets larger and larger
this approximation gets closer and closer to $(3/2)(1-0)=3/2$, so that
$3/2$ is the exact distance traveled during one second, and the final
position is $11.5$.

So for $t=1$, at least, this rather cumbersome approach gives the same
answer as the first approach. But really there's nothing special about
$t=1$; let's just call it $t$ instead. In this case the approximate
distance traveled during time interval number $i$ is $\ds
3(i-1)(t/n)(t/n)=3(i-1)t^2/n^2$, that is, speed $3(i-1)(t/n)$ times
time $t/n$, and the total distance traveled is approximately
$$
  (0){t\over n}+3(1){t^2\over n^2}+3(2){t^2\over n^2}+
  3(3){t^2\over n^2}+\cdots+3(n-1){t^2\over n^2}.
$$
As before we can simplify this to
$$
  {3t^2\over n^2}(0+1+2+\cdots+(n-1))={3t^2\over n^2}{n^2-n\over2}=
  {3\over2}t^2\left(1-{1\over n}\right).
$$ 
In the limit, as $n$ gets larger, this gets closer and closer to $\ds
(3/2)t^2$ and the approximated position of the object gets closer and
closer to $\ds (3/2)t^2+10$, so the actual position is $\ds
(3/2)t^2+10$, exactly the answer given by the first approach to the
problem.
\end{example}

\exam\relax \label{example:area under line} Find the area under the
curve $y=3x$ between $x=0$ and any positive value $x$. There is here
no obvious analogue to the first approach in the previous example, but
the second approach works fine. (Because the function $y=3x$ is so
simple, there is another approach that works here, but it is even more
limited in potential application than is approach number one.)  How
might we approximate the desired area? We know how to compute areas of
rectangles, so we approximate the area by rectangles. Jumping straight
to the general case, suppose we divide the interval between 0 and $x$
into $n$ equal subintervals, and use a rectangle above each
subinterval to approximate the area under the curve. There are many
ways we might do this, but let's use the height of the curve at the
left endpoint of the subinterval as the height of the rectangle, as in
figure~\xrefn{fig:approximating area by rectangles}. The height of
rectangle number $i$ is then $3(i-1)(x/n)$, the width is $x/n$, and
the area is $\ds 3(i-1)(x^2/n^2)$. The total area of the rectangles is
$$
  (0){x\over n}+3(1){x^2\over n^2}+3(2){x^2\over n^2}+
  3(3){x^2\over n^2}+\cdots+3(n-1){x^2\over n^2}.
$$
By factoring out $\ds 3x^2/n^2$ this simplifies to 
$$
  {3x^2\over n^2}(0+1+2+\cdots+(n-1))={3x^2\over n^2}{n^2-n\over2}=
  {3\over2}x^2\left(1-{1\over n}\right).
$$
As $n$ gets larger this gets closer and closer to $\ds 3x^2/2$, which must
therefore be the true area under the curve.
\end{example}

\figure
\vbox{\beginpicture
\normalgraphs
\ninepoint
\setcoordinatesystem units <0.5truecm,0.2truecm>
\setplotarea x from 0 to 10, y from 0 to 30
\axis left shiftedto x=0 /
\axis bottom shiftedto y=0 /
\put {$\ldots$} at 6.5 8
\setlinear
\plot 0 0 10 30 /
\setdashes <2pt>
\putrule from 1 0 to 1 3
\putrule from 2 0 to 2 6
\putrule from 3 0 to 3 9
\putrule from 4 0 to 4 9
\putrule from 9 0 to 9 27
\putrule from 10 0 to 10 27
\putrule from 1 3 to 2 3
\putrule from 2 6 to 3 6
\putrule from 3 9 to 4 9
\putrule from 9 27 to 10 27
\endpicture}
\figrdef{fig:approximating area by rectangles}
\endfigure{Approximating the area under $y=3x$ with rectangles.}

What you will have noticed, of course, is that while the problem in
the second example appears to be much different than the problem in
the first example, and while the easy approach to problem one does not
appear to apply to problem two, the ``approximation'' approach works
in both, and moreover the {\it calculations are identical.} As we will
see, there are many, many problems that appear much different on the
surface but that turn out to be the same as these problems, in the
sense that when we try to approximate solutions we end up with
mathematics that looks like the two examples, though of course the
function involved will not always be so simple.

Even better, we now see that while the second problem did not appear
to be amenable to approach one, it can in fact be solved in the same
way. The reasoning is this: we know that problem one can be solved
easily by finding a function whose derivative is $3t$. We also know
that mathematically the two problems are the same, because both can be
solved by taking a limit of a sum, and the sums are
identical. Therefore, we don't really need to compute the limit of
either sum because we know that we will get the same answer by
computing a function with the derivative $3t$ or, which is the same
thing, $3x$.

It's true that the first problem had the added complication of the
``10'', and we certainly need to be able to deal with such minor
variations, but that turns out to be quite simple. The lesson then is
this: whenever we can solve a problem by taking the limit of a sum of
a certain form, we can instead of computing the (often nasty) limit
find a new function with a certain derivative.

\begin{exercises}

\begin{exercise} Suppose an object moves in a straight line so that its speed at
time $t$ is given by $v(t)=2t+2$, and that at $t=1$ the object is at
position 5. Find the position of the object at $t=2$.
\begin{answer} 10
\end{answer}\end{exercise}

\begin{exercise} Suppose an object moves in a straight line so that its speed at
time $t$ is given by $\ds v(t)=t^2+2$, and that at $t=0$ the object is at
position 5. Find the position of the object at $t=2$.
\begin{answer} $35/3$
\end{answer}\end{exercise}

\begin{exercise} By a method similar to that in example~\xrefn{example:area
  under line}, find the area under $y=2x$ between $x=0$ and any
  positive value for $x$.
\begin{answer} $\ds x^2$
\end{answer}\end{exercise}

\begin{exercise} By a method similar to that in example~\xrefn{example:area
  under line}, find the area under $y=4x$ between $x=0$ and any
  positive value for $x$.
\begin{answer} $\ds 2x^2$
\end{answer}\end{exercise}

\begin{exercise} By a method similar to that in example~\xrefn{example:area
  under line}, find the area under $y=4x$ between $x=2$ and any
  positive value for $x$ bigger than 2.
\begin{answer} $\ds 2x^2-8$
\end{answer}\end{exercise}

\begin{exercise} By a method similar to that in example~\xrefn{example:area
  under line}, find the area under $y=4x$ between any two positive
  values for $x$, say $a<b$.
\begin{answer} $\ds 2b^2-2a^2$
\end{answer}\end{exercise}

\begin{exercise} Let $\ds f(x)=x^2+3x+2$. Approximate the area under the curve
between $x=0$ and $x=2$ using 4 rectangles and also using 8
rectangles. 
\begin{answer} 4 rectangles: $41/4=10.25$,\hfill\break 
8 rectangles: $183/16= 11.4375$
\end{answer}\end{exercise}

\begin{exercise} Let $\ds f(x)=x^2-2x+3$. Approximate the area under the curve
between $x=1$ and $x=3$ using 4 rectangles. 
\begin{answer} $ 23/4$
\end{answer}\end{exercise}

\end{exercises}


