\section{Powers of sine and cosine}{}{}
\nobreak
Functions consisting of products of the sine and cosine can be
integrated by using substitution and trigonometric identities. These
can sometimes be tedious, but the technique is straightforward. Some
examples will suffice to explain the approach.

\begin{example}
Evaluate $\ds\int \sin^5 x\,dx$.
Rewrite the function:
$$
  \int \sin^5 x\,dx=\int \sin x \sin^4 x\,dx=
  \int \sin x (\sin^2 x)^2\,dx=
  \int \sin x (1-\cos^2 x)^2\,dx.
$$
Now use $u=\cos x$, $du=-\sin x\,dx$:
$$\eqalign{
  \int \sin x (1-\cos^2 x)^2\,dx&=\int -(1-u^2)^2\,du \\
  &=\int -(1-2u^2+u^4)\,du \\
  &=-u+{2\over3}u^3-{1\over5}u^5+C \\
  &=-\cos x+{2\over3}\cos^3 x-{1\over5}\cos^5x+C. \\
}$$
\vskip-10pt\end{example}

\begin{example}
Evaluate $\ds\int \sin^6 x\,dx$.
Use $\ds \sin^2x =(1-\cos(2x))/2$ to
rewrite the function:
$$\eqalign{
  \int \sin^6 x\,dx=\int (\sin^2 x)^3\,dx&=
  \int {(1-\cos 2x)^3\over 8}\,dx \\
  &={1\over 8}\int 1-3\cos 2x+3\cos^2 2x-\cos^3 2x\,dx. \\}
$$
Now we have four integrals to evaluate:
$$\int 1\,dx=x$$
and
$$\int -3\cos 2x\,dx = -{3\over 2}\sin 2x$$
are easy. The $\ds \cos^3 2x$ integral is like the previous example:
$$\eqalign{
  \int -\cos^3 2x\,dx&=\int -\cos 2x\cos^2 2x\,dx \\
  &=\int -\cos 2x(1-\sin^2 2x)\,dx \\
  &=\int -{1\over 2}(1-u^2)\,du \\
  &=-{1\over 2}\left(u-{u^3\over 3}\right) \\
  &=-{1\over 2}\left(\sin 2x-{\sin^3 2x\over 3}\right).}
$$
And finally we use another trigonometric identity,
$\ds \cos^2x=(1+\cos(2x))/2$:
$$
  \int 3\cos^2 2x\,dx=3\int {1+\cos 4x\over 2}\,dx=
  {3\over 2}\left(x+{\sin 4x\over 4}\right).
$$
So at long last we get
$$
  \int \sin^6 x\,dx = {x\over8} -{3\over 16}\sin 2x 
  -{1\over 16}\left(\sin 2x-{\sin^3 2x\over 3}\right)
  +{3\over 16}\left(x+{\sin 4x\over 4}\right)+C.
$$
\vskip-10pt\end{example}

\begin{example}
Evaluate $\ds\int \sin^2x\cos^2x\,dx$. 
Use the formulas
$\ds \sin^2x =(1-\cos(2x))/2$ and $\ds \cos^2x =(1+\cos(2x))/2$ to get:
$$
  \int \sin^2x\cos^2x\,dx=\int {1-\cos(2x)\over2}\cdot
  {1+\cos(2x)\over2}\,dx.
$$
The remainder is left as an exercise.
\end{example}

\begin{exercises}

Find the antiderivatives.

\twocol

\begin{exercise} $\ds\int \sin^2 x\,dx$
\begin{answer} $x/2-\sin(2x)/4+C$
\end{answer}\end{exercise}

\begin{exercise} $\ds\int \sin^3 x\,dx$
\begin{answer} $\ds -\cos x+(\cos^3x)/3+C$
\end{answer}\end{exercise}

\begin{exercise} $\ds\int \sin^4 x\,dx$
\begin{answer} $3x/8-(\sin 2x)/4+(\sin 4x)/32+C$
\end{answer}\end{exercise}

\begin{exercise} $\ds\int \cos^2 x\sin^3 x\,dx$
\begin{answer} $\ds (\cos^5 x)/5-(\cos^3x)/3+C$
\end{answer}\end{exercise}

\begin{exercise} $\ds\int \cos^3 x\,dx$
\begin{answer} $\ds \sin x-(\sin^3x)/3+C$
\end{answer}\end{exercise}

\begin{exercise} $\ds\int \sin^2 x\cos^2 x\,dx$
\begin{answer} $x/8-(\sin 4x)/32+C$
\end{answer}\end{exercise}

\begin{exercise} $\ds\int \cos^3 x \sin^2 x\,dx$
\begin{answer} $\ds (\sin^3x)/3-(\sin^5x)/5+C$
\end{answer}\end{exercise}

\begin{exercise} $\ds\int \sin x (\cos x)^{3/2}\,dx$
\begin{answer} $\ds -2(\cos x)^{5/2}/5+C$
\end{answer}\end{exercise}

\begin{exercise} $\ds\int \sec^2 x\csc^2 x\,dx$
\begin{answer} $\tan x-\cot x+C$
\end{answer}\end{exercise}

\begin{exercise} $\ds\int \tan^3x \sec x\,dx$
\begin{answer} $\ds (\sec^3x)/3-\sec x+C$
\end{answer}\end{exercise}

\endtwocol

\end{exercises}

