\section{Limits and Continuity}{}{}

To develop calculus for functions of one variable, we needed to make
sense of the concept of a limit, which we needed to understand
continuous functions and to define the derivative. Limits involving
functions of two variables can be considerably more difficult to deal
with; fortunately, most of the functions we encounter are fairly easy
to understand.

The potential difficulty is largely due to the fact that there are
many ways to ``approach'' a point in the $x$-$y$ plane. If we want to
say that $\ds\lim_{(x,y)\to(a,b)}f(x,y)=L$, we need to capture the
idea that as $(x,y)$ gets close to $(a,b)$ then $f(x,y)$ gets close to
$L$. For functions of one variable, $f(x)$, there are only two ways
that $x$ can approach $a$: from the left or right. But there are an
infinite number of ways to approach $(a,b)$: along any one of an
infinite number of lines, or an infinite number of parabolas, or an
infinite number of sine curves, and so on. We might hope that it's
really not so bad---suppose, for example, that along every possible
line through $(a,b)$ the value of $f(x,y)$ gets close to $L$; surely
this means that ``$f(x,y)$ approaches $L$ as $(x,y)$ approaches
$(a,b)$''. Sadly, no.

\figure
\vbox{\beginpicture
\normalgraphs
\ninepoint
\setcoordinatesystem units <3truecm,3truecm>
\setplotarea x from 0 to 1.1, y from 0 to 1.1
\put {\hbox{\epsfxsize12cm\epsfbox{weird_limit.eps}}} at 0 0
\endpicture}
\figrdef{fig:weird limit}
\endfigure{$\ds f(x,y)={xy^2\over x^2+y^4}$
(\expandafter\url\expandafter{\liveurl weird_limit.html}%
AP\endurl)}

\begin{example} Consider $f(x,y)=xy^2/(x^2+y^4)$. When
$x=0$ or $y=0$, $f(x,y)$ is 0, so the limit of $f(x,y)$ approaching the
origin along either the $x$ or $y$ axis is 0. Moreover, along the line
$y=mx$, $f(x,y)=m^2x^3/(x^2+m^4x^4)$. As $x$ approaches 0 this expression
approaches 0 as well. So along every line through the origin $f(x,y)$
approaches 0. Now suppose we approach the origin along $x=y^2$. Then 
$$f(x,y)={y^2y^2\over y^4+y^4}={y^4\over2y^4}={1\over2},$$
so the limit is $1/2$. Looking at figure~\xrefn{fig:weird limit}, it
is apparent that there is a ridge above $x=y^2$. Approaching the
origin along a straight line, we go over the ridge and then drop down
toward 0, but approaching along the ridge the height is a constant
$1/2$. 
\end{example}
\label{exam:weird limit}

Fortunately, we can define the concept of limit without needing to
specify how a particular point is approached---indeed, in 
definition~\xrefn{def:limit}, we didn't need the concept of
``approach.'' Roughly, that definition says that when $x$ is 
close to $a$ then $f(x)$ is close to $L$; there is no mention of
``how'' we get close to $a$. We can adapt that definition to two
variables quite easily:

\begin{definition} (Limit) Suppose $f(x,y)$ is a function. We say that 
$$\lim_{(x,y)\to
  (a,b)}f(x,y)=L$$
 if for every $\epsilon>0$ there is a $\delta > 0$ so that
  whenever $0 < \sqrt{(x-a)^2+(y-b)^2} < \delta$, $|f(x,y)-L|<\epsilon$.
\end{definition}

This says that we can make $|f(x,y)-L|<\epsilon$, no matter how small
$\epsilon$ is, by making the distance from $(x,y)$ to $(a,b)$ ``small
enough''. 

\begin{example} We show that $\ds \lim_{(x,y)\to(0,0)}{3x^2y\over
  x^2+y^2}=0$. Suppose $\epsilon>0$. Then
$$\left|{3x^2y\over x^2+y^2}\right|={x^2\over x^2+y^2}3|y|.$$
Note that $x^2/(x^2+y^2)\le1$ and
$|y|=\sqrt{y^2}\le\sqrt{x^2+y^2}<\delta$. So
$${x^2\over x^2+y^2}3|y|<1\cdot 3\cdot \delta.$$
We want to force this to be less than $\epsilon$ by picking $\delta$
``small enough.'' If we choose $\delta=\epsilon/3$ then
$$\left|{3x^2y\over x^2+y^2}\right|< 1\cdot 3\cdot{\epsilon\over3}=
\epsilon.$$
\end{example}

Recall that a function $f(x)$ is continuous at $x=a$ if 
$\ds\lim_{x\to a}f(x)=f(a)$; roughly this says that there is no
``hole'' or ``jump'' at $x=a$. We can say exactly the same thing about
a function of two variables: $f(x,y)$ is continuous at $(a,b)$ if 
$\ds\lim_{(x,y)\to (a,b)}f(x,y)=f(a,b)$.

\begin{example} The function $f(x,y)=3x^2y/(x^2+y^2)$ is not continuous at $(0,0)$,
because $f(0,0)$ is not defined. However, we know that 
$\ds \lim_{(x,y)\to(0,0)}f(x,y)=0$, so we can easily ``fix'' the
problem, by extending the definition of $f$ so that $f(0,0)=0$.
This surface is shown in figure~\xrefn{fig:removable discontinuity}.
\end{example}

\figure
\vbox{\beginpicture
\normalgraphs
\ninepoint
\setcoordinatesystem units <3truecm,3truecm>
\setplotarea x from 0 to 1.1, y from 0 to 1.1
\put {\hbox{\epsfxsize8cm\epsfbox{removable_discont.eps}}} at 0 0
\endpicture}
\figrdef{fig:removable discontinuity}
\endfigure{$\ds f(x,y)={3x^2y\over x^2+y^2}$
(\expandafter\url\expandafter{\liveurl removable_discont.html}%
AP\endurl)}

Note that in contrast to this example we cannot fix
example~\xrefn{exam:weird limit} at $(0,0)$ because the limit does not
exist. No matter what value we try to assign to $f$ at $(0,0)$ the
surface will have a ``jump'' there. 

Fortunately, the functions we will examine will typically be
continuous almost everywhere. Usually this follows easily from the
fact that closely related functions of one variable are continuous.
As with single variable functions, two classes of common functions are
particularly useful and easy to describe. A
polynomial\index{polynomial!of two variables} in two
variables is a sum of terms of the form $ax^my^n$, where $a$ is a real
number and $m$ and $n$ are non-negative integers. A rational
function\index{function!of two variables}
is a quotient of polynomials.

\begin{theorem} Polynomials are continuous everywhere. Rational functions are
continuous everywhere they are defined.
\end{theorem}

\begin{exercises}

Determine whether each limit exists. If it does, find the limit
and prove that it is the limit; if it does not, explain how you know.

\begin{exercise} $\ds\lim_{(x,y)\to(0,0)}{x^2\over x^2+y^2}$
\begin{answer} No limit; use $x=0$ and $y=0$.
\end{answer}\end{exercise}

\begin{exercise} $\ds\lim_{(x,y)\to(0,0)}{xy\over x^2+y^2}$
\begin{answer} No limit; use $x=0$ and $x=y$.
\end{answer}\end{exercise}

\begin{exercise} $\ds\lim_{(x,y)\to(0,0)}{xy\over 2x^2+y^2}$
\begin{answer} No limit; use $x=0$ and $x=y$.
\end{answer}\end{exercise}

\begin{exercise} $\ds\lim_{(x,y)\to(0,0)}{x^4-y^4\over x^2+y^2}$
\begin{answer} Limit is zero.
\end{answer}\end{exercise}

\begin{exercise} $\ds\lim_{(x,y)\to(0,0)}{\sin(x^2+y^2)\over x^2+y^2}$
\begin{answer} Limit is 1.
\end{answer}\end{exercise}

\begin{exercise} $\ds\lim_{(x,y)\to(0,0)}{xy\over \sqrt{2x^2+y^2}}$
\begin{answer} Limit is zero.
\end{answer}\end{exercise}

\begin{exercise} $\ds\lim_{(x,y)\to(0,0)} {e^{-x^2-y^2}-1\over x^2+y^2}$
\begin{answer} Limit is $-1$.
\end{answer}\end{exercise}

\begin{exercise} $\ds\lim_{(x,y)\to(0,0)}{x^3+y^3\over x^2+y^2}$
\begin{answer} Limit is zero.
\end{answer}\end{exercise}

\begin{exercise} $\ds\lim_{(x,y)\to(0,0)}{x^2 + \sin^2 y\over 2x^2+y^2}$
\begin{answer} No limit; use $x=0$ and $y=0$.
\end{answer}\end{exercise}

\begin{exercise} $\ds\lim_{(x,y)\to(1,0)}{(x-1)^2\ln x\over(x-1)^2+y^2}$
\begin{answer} Limit is zero.
\end{answer}\end{exercise}

\begin{exercise} $\ds\lim_{(x,y)\to(1,-1)}{3x+4y}$
\begin{answer} Limit is $-1$.
\end{answer}\end{exercise}

\begin{exercise} $\ds\lim_{(x,y)\to(0,0)}{4x^2y\over x^2+y^2}$
\begin{answer} Limit is zero.
\end{answer}\end{exercise}

\begin{exercise} Does the function $\ds f(x,y)={x-y\over 1+x+y}$ 
have any discontinuities?  What about 
$\ds f(x,y)={x-y\over 1+x^2+y^2}$?  Explain.

\end{exercises}

