\section{Maxima and Minima}{}{}
\nobreak
A {\dfont local maximum\index{local maximum} point\/} on a function is a
point $(x,y)$ on the graph of the function whose $y$ coordinate is
larger than all other $y$ coordinates on the graph at points ``close
to'' $(x,y)$. More precisely, $(x,f(x))$ is a local maximum if there
is an interval $(a,b)$ with $a<x<b$ and $f(x)\ge f(z)$ for every $z$
in $(a,b)$. Similarly, $(x,y)$ is a {\dfont local minimum\index{local
minimum} point\/} if it has locally the smallest $y$ coordinate. Again
being more precise: $(x,f(x))$ is a local minimum if there
is an interval $(a,b)$ with $a<x<b$ and $f(x)\le f(z)$ for every $z$
in $(a,b)$. A 
{\dfont local extremum\index{local
extremum}\/} is either a local minimum or a local maximum.

Local maximum and minimum points are quite distinctive on the graph of
a function, and are therefore useful in understanding the shape of the
graph. In many applied problems we want to find the largest or
smallest value that a function achieves (for example, we might want
to find the minimum cost at which some task can be performed) and so
identifying maximum and minimum points will be useful for applied
problems as well. Some examples of local maximum and minimum points
are shown in figure~\xrefn{fig:max and min points}.

% BADBAD
% \figure
% \vbox{\beginpicture
% \normalgraphs
% \ninepoint
% \setcoordinatesystem units <1.5truecm,1.5truecm>
% \setplotarea x from -1.5 to 1.5, y from -1.5 to 1.5
% \axis left shiftedto x=0 /
% \axis bottom shiftedto y=0 /
% \setquadratic
% \plot -1.400 -1.344 -1.330 -1.023 -1.260 -0.740 -1.190 -0.495 -1.120 -0.285 
% -1.050 -0.108 -0.980 0.039 -0.910 0.156 -0.840 0.247 -0.770 0.313 
% -0.700 0.357 -0.630 0.380 -0.560 0.384 -0.490 0.372 -0.420 0.346 
% -0.350 0.307 -0.280 0.258 -0.210 0.201 -0.140 0.137 -0.070 0.070 
% 0.000 0.000 0.070 -0.070 0.140 -0.137 0.210 -0.201 0.280 -0.258 
% 0.350 -0.307 0.420 -0.346 0.490 -0.372 0.560 -0.384 0.630 -0.380 
% 0.700 -0.357 0.770 -0.313 0.840 -0.247 0.910 -0.156 0.980 -0.039 
% 1.050 0.108 1.120 0.285 1.190 0.495 1.260 0.740 1.330 1.023 
% 1.400 1.344 /
% \put {$\bullet$} at -0.5512 0.3752
% \put {$\bullet$} at 0.5512 -0.3752
% \put {$A$} [b] <0pt,4pt> at -0.5512 0.3752
% \put {$B$} [t] <0pt,-4pt> at 0.5512 -0.3752
% \setcoordinatesystem units <1.5truecm,1.5truecm> point at -4 0
% \setplotarea x from -1.5 to 1.5, y from -1.5 to 1.5
% \axis left shiftedto x=0 /
% \axis bottom shiftedto y=0 /
% \setlinear
% \plot -1.5 -1 -1 0.5 -0.25 -0.25 0.5 1.4 1.5 -0.5 /
% \multiput {$\bullet$} at -1 0.5 -0.25 -0.25 0.5 1.4 /
% \multiput {$A$} [b] <0pt,4pt> at -1 0.5 0.5 1.4 /
% \put {$B$} [t] <0pt,-4pt> at -0.25 -0.25 
% \endpicture}
% \figrdef{fig:max and min points}
% \endfigure{Some local maximum points ($A$) and minimum points ($B$).}

If $(x,f(x))$ is a point where $f(x)$ reaches a local maximum or minimum,
and if the derivative of $f$ exists at $x$, then the graph has a
tangent line and the tangent line must be horizontal. This is
important enough to state as a theorem, though we will not prove it.

\begin{theorem} (Fermat's Theorem) If $f(x)$ has a local extremum at $x=a$ and
$f$ is differentiable at $a$, then $f'(a)=0$.
\end{theorem}
\index{Fermat's Theorem}

Thus, the only
points at which a function can have a local maximum or minimum are
points at which the derivative is zero, as in the left hand graph in
figure~\xrefn{fig:max and min points},
or the derivative is undefined, as in the right hand graph. Any value
of $x$ for which $f'(x)$ is zero or undefined is called a
{\dfont critical\index{critical value}
value\/} for $f$.
When looking for local maximum and minimum points, you are likely to
make two sorts of mistakes: You may forget that a maximum or minimum
can occur where the derivative does not exist, and so forget to check
whether the derivative exists everywhere. You might also assume that
any place that the derivative is zero is a local maximum or minimum
point, but this is not true. A portion of the graph of $\ds f(x)=x^3$ is
shown in figure~\xrefn{fig:non extremum}. The derivative of $f$ is
$f'(x)=3x^2$, and $f'(0)=0$, but there is neither a maximum nor
minimum at $(0,0)$.

% BADBAD
% \figure
% \vbox{\beginpicture
% \normalgraphs
% \ninepoint
% \setcoordinatesystem units <1.5truecm,0.75truecm>
% \setplotarea x from -1.5 to 1.5, y from -3.4 to 3.4
% \axis left shiftedto x=0 /
% \axis bottom shiftedto y=0 /
% \setquadratic
% \plot -1.500 -3.375 -1.425 -2.894 -1.350 -2.460 -1.275 -2.073 -1.200 -1.728 
% -1.125 -1.424 -1.050 -1.158 -0.975 -0.927 -0.900 -0.729 -0.825 -0.562 
% -0.750 -0.422 -0.675 -0.308 -0.600 -0.216 -0.525 -0.145 -0.450 -0.091 
% -0.375 -0.053 -0.300 -0.027 -0.225 -0.011 -0.150 -0.003 -0.075 -0.000 
% 0.000 0.000 0.075 0.000 0.150 0.003 0.225 0.011 0.300 0.027 
% 0.375 0.053 0.450 0.091 0.525 0.145 0.600 0.216 0.675 0.308 
% 0.750 0.422 0.825 0.562 0.900 0.729 0.975 0.927 1.050 1.158 
% 1.125 1.424 1.200 1.728 1.275 2.073 1.350 2.460 1.425 2.894 
% 1.500 3.375 /
% \endpicture}
% \figrdef{fig:non extremum}
% \endfigure{No maximum or minimum even though the derivative is zero.}

Since the derivative is zero or undefined at both local maximum and
local minimum points, we need a way to determine which, if either,
actually occurs. The most
elementary approach, but one that is often tedious or difficult, is to
test directly whether the $y$ coordinates ``near'' the potential
maximum or minimum are above or below the $y$ coordinate at the point
of interest. Of course, there are too many points ``near'' the point
to test, but a little thought shows we need only test two provided we
know that $f$ is continuous (recall that this means that the graph of
$f$ has no jumps or gaps).

Suppose, for example, that we have identified three points at which
$f'$ is zero or nonexistent: $\ds (x_1,y_1)$, $\ds (x_2,y_2)$, $\ds (x_3,y_3)$,
and $\ds x_1<x_2<x_3$ (see figure~\xrefn{fig:testing for max and
min}). Suppose that we compute the value of $f(a)$ for $\ds x_1<a<x_2$, and
that $\ds f(a)<f(x_2)$. What can we say about the graph between $a$ and
$\ds x_2$? Could there be a point $\ds (b,f(b))$, $\ds a<b<x_2$ with
$\ds f(b)>f(x_2)$? No: if there were, the graph would go up from
$(a,f(a))$ to $(b,f(b))$ then down to $\ds (x_2,f(x_2))$ and somewhere in
between would have a local maximum point. (This is not obvious; it is
a result of the Extreme Value Theorem, theorem~\xrefn{thm:evt}.)
But at that local maximum
point the derivative of $f$ would be zero or nonexistent, yet we
already know that the derivative is zero or nonexistent only at $\ds x_1$,
$\ds x_2$, and $\ds x_3$. The upshot is that one computation tells us that
$\ds (x_2,f(x_2))$ has the largest $y$ coordinate of any point on the
graph near $\ds x_2$ and to the left of $\ds x_2$. We can perform the same
test on the right. If we find that on both sides of $\ds x_2$ the values
are smaller, then there must be a local maximum at $\ds (x_2,f(x_2))$; if
we find that on both sides of $\ds x_2$ the values are larger, then there
must be a local minimum at $\ds (x_2,f(x_2))$; if we find one of each,
then there is neither a local maximum or minimum at $\ds x_2$.

% BADBAD
% \figure
% \vbox{\beginpicture
% \normalgraphs
% \ninepoint
% \setcoordinatesystem units <1.5truecm,1truecm>
% \setplotarea x from -1.5 to 1.5, y from -1.5 to 1.5
% \axis left shiftedto x=0 /
% \axis bottom shiftedto y=0 ticks withvalues 
% {$x_1$} {$a$} {$b$} {$x_2$} {$x_3$} / at
% -1.3 -1 -0.5 0.5 1 / /
% \multiput {$\bullet$} at -1.3 0.5 -1 0.8 0.5 1.3 1 0.2 /
% \setquadratic
% \endpicture}
% \figrdef{fig:testing for max and min}
% \endfigure{Testing for a maximum or minimum.}

It is not always easy to compute the value of a function at a
particular point. The task is made easier by the availability of
calculators and computers, but they have their own drawbacks---they do
not always allow us to distinguish between values that are very close
together. Nevertheless, because this method is conceptually simple and
sometimes easy to perform, you should always consider it.

\begin{example}
Find all local maximum and minimum points for the function 
$\ds f(x)=x^3-x$. The derivative is $\ds f'(x)=3x^2-1$. This is defined
everywhere and is zero at $\ds x=\pm \sqrt{3}/3$. Looking first at
$\ds x=\sqrt{3}/3$, we see that $\ds f(\sqrt{3}/3)=-2\sqrt{3}/9$. Now we test
two points on either side of 
$\ds x=\sqrt{3}/3$, making sure that neither is farther away than
the nearest critical value; since $\ds \sqrt{3}<3$, $\ds \sqrt{3}/3<1$ and
we can use $x=0$ and $x=1$. Since
$\ds f(0)=0>-2\sqrt{3}/9$
and $\ds f(1)=0>-2\sqrt{3}/9$, there must be a local minimum at 
$\ds x=\sqrt{3}/3$. For $\ds x=-\sqrt{3}/3$, we see that
$\ds f(-\sqrt{3}/3)=2\sqrt{3}/9$. This time we can use $x=0$ and $x=-1$,
and we find that $\ds f(-1)=f(0)=0< 2\sqrt{3}/9$, so there must be a local
maximum at $\ds x=-\sqrt{3}/3$.
\end{example}
\label{exam:simple cubic}

Of course this example is made very simple by our choice of points to
test, namely $x=-1$, $0$, $1$. We could have used other values, say
$-5/4$, $1/3$, and $3/4$, but this would have made the calculations
considerably more tedious.

\begin{example}
\label{example:max and min}
Find all local maximum and minimum points for 
$f(x)=\sin x+\cos x$. The derivative is $f'(x)=\cos x-\sin x$. This is
always defined and is zero whenever $\cos x=\sin x$. Recalling that
the $\cos x$ and $\sin x$ are the $x$ and $y$ coordinates of points on
a unit circle, we see that $\cos x=\sin x$ when $x$ is $\pi/4$, 
$\pi/4\pm\pi$, $\pi/4\pm2\pi$, $\pi/4\pm3\pi$, etc. Since both sine
and cosine have a period of $2\pi$, we need only determine the status
of $x=\pi/4$ and $x=5\pi/4$. We can use $0$ and $\pi/2$ to test the
critical value $x= \pi/4$. 
We find that $\ds f(\pi/4)=\sqrt{2}$, $\ds f(0)=1<\sqrt{2}$ and $\ds f(\pi/2)=1$,
so there is a local maximum when $x=\pi/4$ and also when
$x=\pi/4\pm2\pi$, $\pi/4\pm4\pi$, etc. We can summarize this more
neatly by saying that there are local maxima at $\pi/4\pm 2k\pi$ for
every integer $k$.

We use $\pi$ and $2\pi$ to test the critical value $x=5\pi/4$. The
relevant values are $\ds f(5\pi/4)=-\sqrt2$, $\ds f(\pi)=-1>-\sqrt2$,
$\ds f(2\pi)=1>-\sqrt2$, so there is a local minimum at $x=5\pi/4$,
$5\pi/4\pm2\pi$, $5\pi/4\pm4\pi$, etc. More succinctly, there are
local minima at $5\pi/4\pm 2k\pi$ for
every integer $k$.
\end{example}

\begin{exercises} In problems 1--12, find all local maximum and minimum
points $(x,y)$ by the method of this section.

\twocol
\begin{exercise} $\ds y=x^2-x$ 
\begin{answer} min at $x=1/2$
\end{answer}\end{exercise}

\begin{exercise} $\ds y=2+3x-x^3$ 
\begin{answer} min at $x=-1$, max at $x=1$
\end{answer}\end{exercise}

\begin{exercise} $\ds y=x^3-9x^2+24x$
\begin{answer} max at $x=2$, min at $x=4$
\end{answer}\end{exercise}

\begin{exercise} $\ds y=x^4-2x^2+3$ 
\begin{answer} min at $x=\pm 1$, max at $x=0$.
\end{answer}\end{exercise}

\begin{exercise} $\ds y=3x^4-4x^3$
\begin{answer} min at $x=1$
\end{answer}\end{exercise}

\begin{exercise} $\ds y=(x^2-1)/x$
\begin{answer} none
\end{answer}\end{exercise}

\begin{exercise} $\ds y=3x^2-(1/x^2)$ 
\begin{answer} none
\end{answer}\end{exercise}

\begin{exercise} $y=\cos(2x)-x$ 
\begin{answer} min at $x=7\pi/12+k\pi$, max at $x=-\pi/12+k\pi$, for integer $k$.
\end{answer}\end{exercise}

\begin{exercise} $\ds f(x) = \begin{cases} x-1 & x < 2  \\
x^2 & x\geq 2 \end{cases}$
\begin{answer} none
\end{answer}\end{exercise}

 \begin{exercise} $\ds f(x) = \begin{cases} x-3 & x < 3  \\
x^3  & 3\leq x \leq 5 \\
1/x  & x>5 \end{cases}$
\begin{answer} local max at $x=5$
\end{answer}\end{exercise}

\begin{exercise} $\ds f(x) = x^2 - 98x + 4$
%(Hint: Complete the square.)
\begin{answer} local min at $x=49$
\end{answer}\end{exercise}

\begin{exercise} $\ds f(x) =\begin{cases} -2 & x = 0  \\
1/x^2 & x \neq 0 \end{cases}$
\begin{answer} local min at $x=0$
\end{answer}\end{exercise}

\endtwocol
\bsk
\begin{exercise}  For any real number $x$ there is a unique
  integer $n$ such that $n \leq x < n +1$, and the greatest
  integer\index{greatest integer} function is defined as $\ds\lfloor
  x\rfloor = n$. Where
  are the critical values of the greatest integer function?  Which are
  local maxima and which are local minima?
\end{exercise}

\begin{exercise} Explain why the function $f(x) =1/x$ has no local
maxima or minima.
\end{exercise}

\begin{exercise} How many critical points can a quadratic polynomial function have?
\begin{answer} one
\end{answer}\end{exercise}

\begin{exercise} Show that a cubic polynomial can have at most two critical
points. Give examples to show that a cubic polynomial can have zero,
one, or two critical points.
\end{exercise}

\begin{exercise} Explore the family of functions $\ds f(x) = x^3 + cx +1$ where $c$
 is a constant.  How many and what types of local extremes are there?
 Your answer should depend on the value of $c$, that is, different
 values of $c$ will give different answers.
\end{exercise}

\begin{exercise} We generalize the preceding two questions. Let $n$ be a
positive integer and let $f$ be a polynomial of degree $n$. How many
critical points can $f$ have? (Hint: Recall the {\dfont Fundamental
  Theorem of Algebra\index{Fundamental Theorem of Algebra}}, 
which says that a polynomial of degree $n$ has
  at most $n$ roots.)
\end{exercise}

\end{exercises}

