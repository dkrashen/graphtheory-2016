\section{The Product Rule}{}{}
\index{product rule}

Consider the product of two simple functions, say
$\ds f(x)=(x^2+1)(x^3-3x)$. An obvious guess for the derivative of $f$ is
the product of the derivatives of the constituent functions:
$\ds (2x)(3x^2-3)=6x^3-6x$. Is this correct? We can easily check, by
rewriting $f$ and doing the calculation in a way that is known to
work. First, $\ds f(x)=x^5-3x^3+x^3-3x=x^5-2x^3-3x$, and then
$\ds f'(x)=5x^4-6x^2-3$. Not even close! What went ``wrong''? Well,
nothing really, except the guess was wrong. 

So the derivative of $f(x)g(x)$ is NOT as simple as
$f'(x)g'(x)$. Surely there is some rule for such a situation? There
is, and it is instructive to ``discover'' it by trying to do the
general calculation even without knowing the answer in advance.
\begin{align*}
{d\over dx}(&f(x)g(x)) = \lim_{\Delta x \to0} {f(x+\Delta
  x)g(x+\Delta x) - f(x)g(x)\over \Delta x} \\
&=\lim_{\Delta x \to0} {f(x+\Delta
  x)g(x+\Delta x)-f(x+\Delta x)g(x) + f(x+\Delta x)g(x)- f(x)g(x)\over
  \Delta x} \\ 
&=\lim_{\Delta x \to0} {f(x+\Delta
  x)g(x+\Delta x)-f(x+\Delta x)g(x)\over \Delta x} + 
\lim_{\Delta x \to0} {f(x+\Delta x)g(x)- f(x)g(x)\over
  \Delta x} \\ 
&=\lim_{\Delta x \to0} f(x+\Delta x){
 g(x+\Delta x)-g(x)\over \Delta x} + 
\lim_{\Delta x \to0} {f(x+\Delta x)- f(x)\over
  \Delta x}g(x) \\ 
&=f(x)g'(x) + f'(x)g(x) \\
\end{align*}
A couple of items here need discussion. First, we used a standard
trick, ``add and subtract the same thing'', to transform what we had
into a more useful form. After some rewriting, we realize that we have
two limits that produce $f'(x)$ and $g'(x)$. Of course, $f'(x)$ and
$g'(x)$ must actually exist for this to make sense.
We also replaced
$\ds \lim_{\Delta x\to0}f(x+\Delta x)$ with $f(x)$---why is this justified? 

What we really need to know here is that $\ds \lim_{\Delta x\to
  0}f(x+\Delta x)=f(x)$, or in the language of
section~\xrefn{sec:adjectives for functions}, that $f$ is continuous
at $x$.  We already know that $f'(x)$ exists (or the whole approach,
writing the derivative of $fg$ in terms of $f'$ and $g'$, doesn't make
sense). This turns out to imply that $f$ is continuous as well. Here's
why:
\begin{align*}
\lim_{\Delta x\to 0} f(x+\Delta x) &= \lim_{\Delta x\to 0} (f(x+\Delta
x) -f(x) + f(x)) \\
&= \lim_{\Delta x\to 0} {f(x+\Delta x) -f(x)\over \Delta x}\Delta x +
\lim_{\Delta x\to 0} f(x) \\
&=f'(x)\cdot 0 + f(x) = f(x) \\
\end{align*}

% \figure
% \vbox{\beginpicture
% \normalgraphs
% \ninepoint
% \setcoordinatesystem units <1truecm,1truecm>
% \setplotarea x from -2 to 2, y from -2 to 2
% \axis left shiftedto x=0  /
% \axis bottom shiftedto y=0 /
% \linethickness 1pt
% \put {$\bullet$} at 2 2
% \put {$\bullet$} at 1 1
% \put {$\bullet$} at 0 0
% \put {$\bullet$} at -1 -1
% \put {$\bullet$} at -2 -2
% \put {$\circ$} at 1 0
% \put {$\circ$} at 2 1
% \put {$\circ$} at 0 -1
% \put {$\circ$} at -1 -2
% \putrule from -2 -2 to -1 -2
% \putrule from -1 -1 to 0 -1
% \putrule from 0 0 to 1 0
% \putrule from 1 1 to 2 1
% \endpicture}
% \figrdef{fig:step function}
% \endfigure{The greatest integer function.}


To summarize: the product\index{product rule} rule says that
$${d\over dx}(f(x)g(x)) = f(x)g'(x) + f'(x)g(x).
$$

Returning to the example we started with, let $\ds f(x)=(x^2+1)(x^3-3x)$.
Then $\ds f'(x)=(x^2+1)(3x^2-3)+(2x)(x^3-3x)=3x^4-3x^2+3x^2-3+2x^4-6x^2=
5x^4-6x^2-3$, as before. In this case it is probably simpler to
multiply $f(x)$ out first, then compute the derivative; here's an
example for which we really need the product rule.

\begin{example} Compute the derivative of $\ds f(x)=x^2\sqrt{625-x^2}$.  We have
already computed $\ds {d\over
  dx}\sqrt{625-x^2}={-x\over\sqrt{625-x^2}}$.  Now
$$f'(x)=x^2{-x\over\sqrt{625-x^2}}+2x\sqrt{625-x^2}=
{-x^3+2x(625-x^2)\over \sqrt{625-x^2}}=
{-3x^3+1250x\over \sqrt{625-x^2}}.
$$
\vskip-10pt\end{example}

\begin{exercises}

In 1--4, find the derivatives of the functions using the product rule.

\begin{exercise} $\ds x^3(x^3-5x+10)$
\begin{answer} $\ds 3x^2(x^3-5x+10)+x^3(3x^2-5)$
\end{answer}\end{exercise}

\begin{exercise} $\ds (x^2+5x-3)(x^5-6x^3+3x^2-7x+1)$
\begin{answer} $\ds (x^2+5x-3)(5x^4-18x^2+6x-7)+(2x+5)(x^5-6x^3+3x^2-7x+1)$
\end{answer}\end{exercise}

\begin{exercise} $\ds \sqrt{x}\sqrt{625-x^2}$
\begin{answer} $\ds \ds{\sqrt{625-x^2}\over 2\sqrt{x}}-{x\sqrt{x}\over\sqrt{625-x^2}}$
\end{answer}\end{exercise}

\begin{exercise} $\displaystyle {\sqrt{625-x^2}\over x^{20}}$
\begin{answer} $\ds{-1\over x^{19}\sqrt{625-x^2}}-{20\sqrt{625-x^2}\over x^{21}}$
\end{answer}\end{exercise}

\begin{exercise} Use the product rule to compute the derivative of $\ds f(x)=(2x-3)^2$.
 Sketch the function.  Find an equation of the tangent line to the curve at
 $x=2$.  Sketch the tangent line at $x=2$.
\begin{answer} $f'=4(2x-3)$, $y=4x-7$
\end{answer}\end{exercise}

\begin{exercise}
Suppose that $f$, $g$, and $h$ are differentiable functions.
Show that $(fgh)'(x) = f'(x) g(x)h(x) + f(x)g'(x) h(x) + f(x) g(x)
h'(x)$.
\end{exercise}

\begin{exercise}
State and prove a rule to compute $(fghi)'(x)$, 
similar to the rule in the previous problem.
\end{exercise}

\begin{remark}{Product notation}
Suppose $\ds f_1 , f_2 , \ldots f_n$ are functions.
The product of all these functions can be written
$$ \prod _{k=1 } ^n f_k.$$
This is similar to the use of $\ds \sum$ to denote a 
sum.
For example,
$$\prod _{k=1 } ^5 f_k =f_1 f_2 f_3 f_4 f_5$$
and
$$
\prod _ {k=1 } ^n k = 1\cdot 2 \cdot \ldots \cdot n = n!.$$
We sometimes use somewhat more complicated conditions; for example
$$\prod _{k=1 , k\neq j } ^n f_k$$
denotes the product of $\ds f_1$ through $\ds f_n$ except for $\ds f_j$.
For example, 
$$\prod _{k=1 , k\neq 4} ^5 x^k = x\cdot x^2 \cdot x^3 \cdot x^5 =
x^{11}.$$
\end{remark}

\begin{exercise}
  The {\dfont generalized product rule\index{product rule!generalized}\/} 
says that if $\ds f_1 , f_2 ,\ldots ,f_n$ are differentiable functions at
  $x$ then
$${d\over dx}\prod _{k=1 } ^n f_k(x) = 
\sum _{j=1 } ^n \left(f'_j (x) \prod _{k=1 , k\neq j} ^n
   f_k (x)\right).$$
Verify that this is the same as your answer to the previous problem
when $n=4$,
and write out what this says when $n=5$.
\end{exercise}

%% Wills
%% \begin{exercise} 
%% The generalized product rule is trivial when $n=1$. The case
%% when $n=2$ is shown in the text and the case when $n=3$ is assigned
%% in an earlier exercise. Use mathematical induction to show that the
%% generalized product rule is true for any positive integer $n$.
%%  \end{exer}

\end{exercises}

