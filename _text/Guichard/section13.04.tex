\section{Motion along a curve}{}{}

We have already seen that if $t$ is time and an object's location is
given by ${\bf r}(t)$, then the derivative ${\bf r}'(t)$ is the
velocity vector\index{velocity vector} ${\bf v}(t)$.
Just as ${\bf v}(t)$ is a vector describing how ${\bf r}(t)$ changes,
so is ${\bf v}'(t)$ a vector describing how ${\bf v}(t)$ changes,
namely, ${\bf a}(t)={\bf v}'(t)={\bf r}''(t)$ is the 
{\dfont acceleration vector\index{acceleration vector}}.

\begin{example} Suppose ${\bf r}(t)=\langle \cos t,\sin t,1\rangle$. Then
${\bf v}(t)=\langle -\sin t,\cos t,0\rangle$ and 
${\bf a}(t)=\langle -\cos t,-\sin t,0\rangle$. This describes the
motion of an object traveling on a circle of radius 1, with constant
$z$ coordinate 1. The velocity vector is of course tangent to the
curve; note that ${\bf a}\cdot{\bf v}=0$, so ${\bf v}$ and ${\bf a}$
are perpendicular. In fact, it is not hard to see that ${\bf a}$
points from the location of the object to the center of the circular
path at $(0,0,1)$.
\end{example}

Recall that the unit tangent vector is given by ${\bf T}(t)=
{\bf v}(t)/|{\bf v}(t)|$, so ${\bf v}=|{\bf v}|{\bf T}$. If we take
the derivative of both sides of this equation we get
$${\bf a}=|{\bf v}|'{\bf T}+|{\bf v}|{\bf T}'.
\eqrdef{eq:acceleration decomposition initial}
\eqno{(\xrefn{eq:acceleration decomposition initial})}$$
Also recall the definition of the curvature,
$\kappa=|{\bf T}'|/|{\bf v}|$, or $|{\bf T}'|=\kappa|{\bf v}|$. Finally,
    recall that we defined the unit normal vector as
${\bf N}={\bf T}'/|{\bf T}'|$, so ${\bf T}'=|{\bf T}'|{\bf N}=
\kappa|{\bf v}|{\bf N}$.
Substituting into equation~\xrefn{eq:acceleration decomposition
  initial} we get
$${\bf a}=|{\bf v}|'{\bf T}+\kappa|{\bf v}|^2{\bf N}.
\eqrdef{eq:acceleration decomposition final}
\eqno{(\xrefn{eq:acceleration decomposition final})}$$
The quantity $|{\bf v}(t)|$ is the speed of the object, often written as
$v(t)$; $|{\bf v}(t)|'$ is the rate at which the speed is changing, or
the scalar acceleration of the object, $a(t)$. Rewriting 
equation~\xrefn{eq:acceleration decomposition final} with these gives
us
$${\bf a}=a{\bf T}+\kappa v^2{\bf N}=
a_{T}{\bf T}+a_{N}{\bf N};$$
$a_T$ is the {\dfont tangential component of
  acceleration\index{acceleration!tangential component}\/} and 
$a_N$ is the {\dfont normal component of
  acceleration\index{acceleration!normal component}}. 
We have already seen that $a_T$ measures how the speed is changing; if
you are riding in a vehicle with large $a_T$ you will feel a force
pulling you into your seat. The other component, $a_N$, measures how
sharply your direction is changing {\it with respect to time}. So it
naturally is related to how sharply the path is curved, measured by
$\kappa$, and also to how fast you are going. Because $a_N$ includes
$v^2$, note that the effect of speed is magnified; doubling your speed
around a curve quadruples the value of $a_N$. You feel the effect of
this as a force pushing you toward the outside of the curve, the
``centrifugal force.''

In practice, if want $a_N$ we would use the formula for $\kappa$:
$$a_N=\kappa |{\bf v}|^2= {|{\bf r}'\times{\bf r}''|\over
|{\bf r}'|^3}|{\bf r}'|^2={|{\bf r}'\times{\bf r}''|\over|{\bf r}'|}.$$
To compute $a_T$ we can project ${\bf a}$ onto ${\bf v}$:
$$a_T={{\bf v}\cdot{\bf a}\over|{\bf v}|}={{\bf r}'\cdot{\bf r}''\over
|{\bf r}'|}.$$

\begin{example} Suppose ${\bf r}=\langle t,t^2,t^3\rangle$. 
Compute ${\bf v}$, ${\bf a}$,
$a_T$, and $a_N$.

Taking derivatives we get
${\bf v}=\langle 1,2t,3t^2\rangle$ 
and ${\bf a}=\langle 0,2,6t\rangle$. Then
$$a_T={4t+18t^3\over \sqrt{1+4t^2+9t^4}}
\quad\hbox{and}\quad
a_N={\sqrt{4+36t^2+36t^4}\over\sqrt{1+4t^2+9t^4}}.$$
\end{example}

\begin{exercises}

\begin{exercise} Let ${\bf r}=\langle \cos t,\sin t,t\rangle$. 
Compute ${\bf v}$, ${\bf a}$,
$a_T$, and $a_N$.
\begin{answer} $\langle -\sin t,\cos t,1\rangle$,
$\langle -\cos t, -\sin t,0\rangle$,
$0$, $1$
\end{answer}\end{exercise}

\begin{exercise} Let ${\bf r}=\langle \cos t,\sin t,t^2\rangle$. 
Compute ${\bf v}$, ${\bf a}$,
$a_T$, and $a_N$.
\begin{answer} $\langle -\sin t,\cos t,2t\rangle$,
$\langle -\cos t, -\sin t,2\rangle$,
$4t/\sqrt{4t^2+1}$, $\sqrt{4t^2+5}/\sqrt{4t^2+1}$
\end{answer}\end{exercise}

\begin{exercise} Let ${\bf r}=\langle \cos t,\sin t,e^t\rangle$. 
Compute ${\bf v}$, ${\bf a}$,
$a_T$, and $a_N$.
\begin{answer} $\langle -\sin t,\cos t,e^t\rangle$,
$\langle -\cos t, -\sin t,e^t\rangle$,
$e^{2t}/\sqrt{e^{2t}+1}$, $\sqrt{2e^{2t}+1}/\sqrt{e^{2t}+1}$
\end{answer}\end{exercise}

\begin{exercise} Let ${\bf r}=\langle e^t,\sin t,e^t\rangle$. 
Compute ${\bf v}$, ${\bf a}$,
$a_T$, and $a_N$.
\begin{answer} $\langle e^t,\cos t,e^t\rangle$,
$\langle e^t, -\sin t,e^t\rangle$,
$(2e^{2t}-\cos t\sin t)/\sqrt{2e^{2t}+\cos^2 t}$, 
$\sqrt{2}e^t|\cos t+\sin t|/\sqrt{2e^{2t}+\cos^2 t}$
\end{answer}\end{exercise}

\begin{exercise} Suppose an object moves so that its acceleration is given by
${\bf a}=\langle -3\cos t,-2\sin t,0\rangle$. At time $t=0$ the object
is at $(3,0,0)$ and its velocity vector is $\langle
0,2,0\rangle$. Find ${\bf v}(t)$ and ${\bf r}(t)$ for the object.
\begin{answer} $\langle -3\sin t,2\cos t,0\rangle$,
$\langle 3\cos t, 2\sin t,0\rangle$
\end{answer}\end{exercise}

\begin{exercise} Suppose an object moves so that its acceleration is given by
${\bf a}=\langle -3\cos t,-2\sin t,0\rangle$. At time $t=0$ the object
is at $(3,0,0)$ and its velocity vector is $\langle
0,2.1,0\rangle$. Find ${\bf v}(t)$ and ${\bf r}(t)$ for the object.
\begin{answer} $\langle -3\sin t,2\cos t+0.1,0\rangle$,
$\langle 3\cos t, 2\sin t+t/10,0\rangle$
\end{answer}\end{exercise}

\begin{exercise} Suppose an object moves so that its acceleration is given by
${\bf a}=\langle -3\cos t,-2\sin t,0\rangle$. At time $t=0$ the object
is at $(3,0,0)$ and its velocity vector is $\langle
0,2,1\rangle$. Find ${\bf v}(t)$ and ${\bf r}(t)$ for the object.
\begin{answer} $\langle -3\sin t,2\cos t,1\rangle$,
$\langle 3\cos t, 2\sin t,t\rangle$
\end{answer}\end{exercise}

\begin{exercise} Suppose an object moves so that its acceleration is given by
${\bf a}=\langle -3\cos t,-2\sin t,0\rangle$. At time $t=0$ the object
is at $(3,0,0)$ and its velocity vector is $\langle
0,2.1,1\rangle$. Find ${\bf v}(t)$ and ${\bf r}(t)$ for the object.
\begin{answer} $\langle -3\sin t,2\cos t+1/10,1\rangle$,
$\langle 3\cos t, 2\sin t+t/10,t\rangle$
\end{answer}\end{exercise}

\begin{exercise} Describe a situation in which the normal component of
acceleration is 0 and the tangential component of acceleration is
non-zero. Is it possible for the tangential component of acceleration
to be 0 while the normal component of acceleration is non-zero? Explain.
Finally, is it possible for an object to move (not be stationary)
so that both the tangential and normal components of acceleration are 0?
Explain.

\end{exercises}
