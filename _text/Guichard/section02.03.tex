\section{Limits}{}{}
\label{sec:limits}

In the previous two sections we computed some quantities of interest
(slope, velocity) by seeing that some expression ``goes to'' or
``approaches'' or ``gets really close to'' a particular value. In the
examples we saw, this idea may have been clear enough, but it is too
fuzzy to rely on in more difficult circumstances. In this section we
will see how to make the idea more precise.

There is an important feature of the examples we have seen. Consider
again the formula
$${-19.6\Delta x-4.9\Delta x^2\over \Delta x}.
$$
We wanted to know what happens to this fraction as ``$\Delta x$ goes
to zero.'' Because we were able to simplify the fraction, it was easy
to see the answer, but it was not quite as simple as ``substituting
zero for $\Delta x$,'' as that would give
$${-19.6\cdot 0 - 4.9\cdot 0\over 0},$$
which is meaningless. The quantity we are really interested in does
not make sense ``at zero,'' and this is why the answer to the original
problem (finding a velocity or a slope) was not immediately obvious. In
other words, we are generally going to want to figure out what a
quantity ``approaches'' in situations where we can't merely plug in a
value. If you would like to think about a hard example (which we will
analyze later) consider what happens to $(\sin x)/x$ as $x$ approaches
zero. 

\begin{example}
Does $\ds \sqrt{x}$ approach 1.41 as $x$ approaches 2? In this case it
is possible to compute the actual value $\ds \sqrt{2}$ to a high precision
to answer the question. But since in general we won't be able to do
that, let's not. We might start by computing $\ds \sqrt{x}$ for values of
$x$ close to 2, as we did in the previous sections. Here are some
values: $\ds \sqrt{2.05} = 1.431782106$,
        $\ds \sqrt{2.04} = 1.428285686$,
        $\ds \sqrt{2.03} = 1.424780685$,
        $\ds \sqrt{2.02} = 1.421267040$,
        $\ds \sqrt{2.01} = 1.417744688$,
        $\ds \sqrt{2.005} = 1.415980226$,
        $\ds \sqrt{2.004} = 1.415627070$,
        $\ds \sqrt{2.003} = 1.415273825$,
        $\ds \sqrt{2.002} = 1.414920492$,
        $\ds \sqrt{2.001} = 1.414567072$.
So it looks at least possible that indeed these values ``approach''
1.41---already $\ds \sqrt{2.001}$ is quite close. If we continue this
process, however, at some point we will appear to ``stall.'' In fact, 
$\ds \sqrt{2}=1.414213562\ldots$, so we will never even get as far as
1.4142, no matter how long we continue the sequence. 
\end{example}

So in a fuzzy,
everyday sort of sense, it is true that $\ds \sqrt{x}$ ``gets close to''
1.41, but it does not ``approach'' 1.41 in the sense we want. To
compute an exact slope or an exact velocity, what we want to know is that
a given quantity becomes ``arbitrarily close'' to a fixed value,
meaning that the first quantity can be made ``as close as we like'' to
the fixed value. Consider again the quantities
$${-19.6\Delta x-4.9\Delta x^2\over \Delta x}=-19.6-4.9\Delta x.
$$
These two quantities are equal as long as $\Delta x$ is not zero; if
$\Delta x$ is zero, the left hand quantity is meaningless, while the
right hand one is $-19.6$. Can we say more than we did before about
why the right hand side ``approaches'' $-19.6$, in the desired sense?
Can we really make it ``as close as we want'' to $-19.6$? Let's try a
test case. Can we make $-19.6-4.9\Delta x$ within one millionth
($0.000001$) of $-19.6$? The values within a millionth of $-19.6$ are
those in the interval $(-19.600001,-19.599999)$. As $\Delta x$
approaches zero, does $-19.6-4.9\Delta x$ eventually reside inside
this interval? If $\Delta x$ is positive, this would require that
$-19.6-4.9\Delta x> -19.600001$. This is something we can manipulate
with a little algebra:
\begin{align*}
-19.6-4.9\Delta x&> -19.600001 \\
-4.9\Delta x&>-0.000001 \\
\Delta x&<-0.000001/-4.9 \\
\Delta x&<0.0000002040816327\ldots \\
\end{align*}
Thus, we can say with certainty that if $\Delta x$ is positive and
less than $0.0000002$, then $\Delta x < 0.0000002040816327\ldots$ and
so $-19.6-4.9\Delta x > -19.600001$. We could do a similar calculation
if $\Delta x$ is negative.

So now we know that we can make $-19.6-4.9\Delta x$ within one
millionth of $-19.6$. But can we make it ``as close as we want''?
In this case, it is quite simple to see that the answer is yes, by
modifying the calculation we've just done. It may be helpful to think
of this as a game. I claim that I can make $-19.6-4.9\Delta x$ as
close as you desire to $-19.6$ by making $\Delta x$ ``close enough''
to zero. So the game is: you give me a number, like $\ds 10^{-6}$, and I
have to come up with a number representing how close $\Delta x$ must
be to zero to guarantee that $-19.6-4.9\Delta x$ is at least as close
to $-19.6$ as you have requested. 

Now if we actually play this game, I could redo the calculation above
for each new number you provide. What I'd like to do is somehow see
that I will always succeed, and even more, I'd like to have a simple
strategy so that I don't have to do all that algebra every time.
A strategy in this case would be a formula that gives me a correct
answer no matter what you specify. So suppose the number you give me
is $\epsilon$. How close does $\Delta x$ have to be to zero to
guarantee that $-19.6-4.9\Delta x$ is in $(-19.6-\epsilon,
-19.6+\epsilon)$? If $\Delta x$ is positive, we need:
\begin{align*}
-19.6-4.9\Delta x&> -19.6-\epsilon \\
-4.9\Delta x&>-\epsilon \\
\Delta x&<-\epsilon/-4.9 \\
\Delta x&<\epsilon/4.9 \\
\end{align*}
So if I pick any number $\delta$ that is less than $\epsilon/4.9$, the
algebra tells me that whenever $\Delta x<\delta$ then 
$\Delta x<\epsilon/4.9$ and so $-19.6-4.9\Delta x$ is within
$\epsilon$ of $-19.6$. (This is exactly what I did in the example:
I picked $\delta = 0.0000002 < 0.0000002040816327\ldots$.)
A similar calculation again works for negative
$\Delta x$.
The important fact is that this is now a completely general result---it
shows that I can always win, no matter what ``move'' you make.

Now we can codify this by giving a precise definition to replace the
fuzzy, ``gets closer and closer'' language we have used so
far. Henceforward, we will say something like ``the limit of
$\ds (-19.6\Delta x-4.9\Delta x^2)/\Delta x$ as $\Delta x$ goes to zero is
$-19.6$,'' and abbreviate this mouthful as
$$\lim_{\Delta x\to 0} {-19.6\Delta x-4.9\Delta x^2\over \Delta x}
= -19.6.
$$
Here is the actual, official definition of ``limit''\index{limit}.

\begin{definition}\label{def:limit} 
Suppose $f$ is a function. We say that the \textbf{limit} of $f(x)$ as
$x$ goes to $a$ is $L$,
\[
\lim_{x\to a}f(x)=L
\] 
if for every $\epsilon>0$ there is a $\delta > 0$ so that whenever $0
< |x-a| < \delta$, $|f(x)-L|<\epsilon$.
\end{definition}


The $\epsilon$ and $\delta$ here play exactly the role they did in the
preceding discussion. The definition says, in a very precise way, that
$f(x)$ can be made as close as desired to $L$ (that's the 
$|f(x)-L|<\epsilon$ part) by making $x$ close enough to $a$ (the 
$0 < |x-a| < \delta$ part). Note that we specifically make no mention
of what must happen if $x=a$, that is, if $|x-a|=0$. This is because
in the cases we are most interested in, substituting $a$ for $x$
doesn't even make sense.

Make sure you are not confused by the names of important
quantities. The generic definition talks about $f(x)$, but the
function and the variable might have other names. In the discussion
above, the function we analyzed was 
$${-19.6\Delta x-4.9\Delta x^2\over \Delta x}.
$$
and the variable of the limit was not $x$ but $\Delta x$. The $x$ was
the variable of the original function; when we were trying to compute
a slope or a velocity, $x$ was essentially a fixed quantity, telling us
at what point we wanted the slope. (In the velocity problem, it was
literally a fixed quantity, as we focused on the time 2.) The quantity
$a$ of the definition in all the examples was zero: we were always
interested in what happened as $\Delta x$ became very close to zero.

Armed with a precise definition, we can now prove that certain
quantities behave in a particular way. The bad news is that even
proofs for simple quantities can be quite tedious and complicated; the
good news is that we rarely need to do such proofs, because most
expressions act the way you would expect, and this can be proved once
and for all.

\exam\relax
\label{exam:epsilon-delta proof}
Let's show carefully that $\ds \lim_{x\to 2} x+4 = 6$. This is not
something we ``need'' to prove, since it is ``obviously'' true. But if
we couldn't prove it using our official definition there would be
something very wrong with the definition.

As is often the case in mathematical proofs, it helps to work
backwards. We want to end up showing that under certain circumstances
$x+4$ is close to 6; precisely, we want to show that
$|x+4-6|<\epsilon$,
or $|x-2|<\epsilon$. Under what circumstances? We want this to be true
whenever $0<|x-2|<\delta$. So the question becomes: can we choose a
value for $\delta$ that guarantees that $0<|x-2|<\delta$ implies
$|x-2|<\epsilon$? Of course: no matter what $\epsilon$ is,
$\delta=\epsilon$ works.
%\end{example}

So it turns out to be very easy to prove something ``obvious,'' which
is nice. It doesn't take long before things get trickier, however.

\begin{example}
It seems clear that $\ds \lim_{x\to 2} x^2=4$. Let's try to prove it. We
will want to be able to show that $\ds |x^2-4|<\epsilon$ whenever 
$0<|x-2|<\delta$, by choosing $\delta$ carefully. Is there any
connection between $|x-2|$ and $\ds |x^2-4|$? Yes, and it's not hard to
spot, but it is not so simple as the previous example. We can write
$\ds |x^2-4|=|(x+2)(x-2)|$. Now when $|x-2|$ is small, part of 
$|(x+2)(x-2)|$ is small, namely $(x-2)$. What about $(x+2)$? If $x$ is
close to 2, $(x+2)$ certainly can't be too big, but we need to somehow
be precise about it. Let's recall the ``game'' version of what is
going on here. You get to pick an $\epsilon$ and I have to pick a
$\delta$ that makes things work out. Presumably it is the really tiny
values of $\epsilon$ I need to worry about, but I have to be prepared
for anything, even an apparently ``bad'' move like $\epsilon=1000$.
I expect that $\epsilon$ is going to be small, and that the
corresponding $\delta$ will be small, certainly less than 1.
If $\delta\le 1$ then
$|x+2|<5$ when $|x-2|<\delta$ (because if $x$ is
within 1 of 2, then $x$ is between 1 and 3 and $x+2$ is between 3 and
5). So then I'd be trying to show that
$|(x+2)(x-2)|<5|x-2|<\epsilon$. So now how can I pick $\delta$ so that 
$|x-2|<\delta$ implies $5|x-2|<\epsilon$? This is easy: use
$\delta=\epsilon/5$, so $5|x-2|<5(\epsilon/5) = \epsilon$. But what if
the $\epsilon$ you choose is not small? If you choose
$\epsilon=1000$, should I pick $\delta=200$? No, to keep things
``sane'' I will never pick a $\delta$ bigger than 1. Here's the final
``game strategy:'' When you pick a value for $\epsilon$ I will pick
$\delta=\epsilon/5$ or $\delta=1$, whichever is smaller. Now when 
$|x-2|<\delta$, I know both that $|x+2|<5$ and that
$|x-2|<\epsilon/5$. Thus $|(x+2)(x-2)|<5(\epsilon/5) = \epsilon$.

This has been a long discussion, but most of it was explanation and
scratch work. If this were written down as a proof, it would be quite
short, like this:

Proof that $\ds \lim_{x\to 2}x^2=4$. Given any $\epsilon$, pick
$\delta=\epsilon/5$ or $\delta=1$, whichever is smaller. Then when
$|x-2|<\delta$, $|x+2|<5$ and
$|x-2|<\epsilon/5$. Hence $\ds |x^2-4|=|(x+2)(x-2)|<5(\epsilon/5) =
\epsilon$. 
\end{example}

It probably seems obvious that $\ds \lim_{x\to2}x^2=4$, and it is worth
examining more closely why it seems obvious. If we write $\ds x^2=x\cdot
x$, and ask what happens when $x$ approaches 2, we might say something
like, ``Well, the first $x$ approaches 2, and the second $x$
approaches 2, so the product must approach $2\cdot2$.'' In fact this is
pretty much right on the money, except for that word ``must.'' Is it
really true that if $x$ approaches $a$ and $y$ approaches $b$ then
$xy$ approaches $ab$? It is, but it is not really obvious, since $x$
and $y$ might be quite complicated. The good news is that we can see
that this is true once and for all, and then we don't have to worry
about it ever again. When we say that $x$ might be ``complicated'' we
really mean that in practice it might be a function. Here is then what
we want to know:

\begin{theorem} Suppose $\ds \lim_{x\to a} f(x)=L$ and $\ds \lim_{x\to a}g(x)=M$. Then
\hfill\break
$\lim_{x\to a} f(x)g(x) = LM$.
\end{theorem}

\begin{proof} We have to use the official definition of limit to make sense
of this. So given any $\epsilon$ we need to find a $\delta$ so that
$0<|x-a|<\delta$ implies $|f(x)g(x)-LM|<\epsilon$. What do we have to
work with? We know that we can make $f(x)$ close to $L$ and $g(x)$
close to $M$, and we have to somehow connect these facts to make
$f(x)g(x)$ close to $LM$.

We use, as is so often the case, a little algebraic
trick: 
\begin{align*}
|f(x)g(x)-LM|&= |f(x)g(x)-f(x)M+f(x)M-LM| \\
&=|f(x)(g(x)-M)+(f(x)-L)M| \\
&\le |f(x)(g(x)-M)|+|(f(x)-L)M| \\
&=|f(x)||g(x)-M|+|f(x)-L||M|.
\end{align*}

This is all straightforward except perhaps for the ``$\le$''. That is
an example of the {\dfont triangle inequality%
\index{triangle inequality}\pagerdef{page:triangle inequality}}, 
which says that if $a$ and $b$ are any real
numbers then $|a+b|\le |a|+|b|$. If you look at a few examples, using
positive and negative numbers in various combinations for $a$ and $b$,
you should quickly understand why this is true; we will not prove it
formally. 

Since $\ds \lim_{x\to a}f(x) =L$, there is a value $\ds \delta_1$ so that
$0<|x-a|<\delta_1$ implies $|f(x)-L|<|\epsilon/(2M)|$, 
This means that $0<|x-a|<\delta_1$ implies
$|f(x)-L||M|< \epsilon/2$. You can see where this is going: if we can
make $|f(x)||g(x)-M|<\epsilon/2$ also, then we'll be done.

We can make $|g(x)-M|$ smaller than any fixed number by making $x$
close enough to $a$; unfortunately, $\epsilon/(2f(x))$ is not a fixed
number, since $x$ is a variable. Here we need another little trick,
just like the one we used in analyzing $x^2$. We can find a $\delta_2$
so that $|x-a|<\delta_2$ implies that $|f(x)-L|<1$, meaning that $L-1
< f(x) < L+1$. This means that $|f(x)|<N$, where $N$ is either $|L-1|$
or $|L+1|$, depending on whether $L$ is negative or positive. The
important point is that $N$ doesn't depend on $x$. Finally, we know that
there is a $\delta_3$ so that $0<|x-a|<\delta_3$ implies
$|g(x)-M|<\epsilon/(2N)$. Now we're ready to put everything
together. Let $\delta$ be the smallest of $\delta_1$, $\delta_2$, and
$\delta_3$. Then $|x-a|<\delta$ implies that
$|f(x)-L|<|\epsilon/(2M)|$, $|f(x)|<N$, and
$|g(x)-M|<\epsilon/(2N)$. Then 
\begin{align*}
|f(x)g(x)-LM|&\le|f(x)||g(x)-M|+|f(x)-L||M| \\
&<N{\epsilon\over 2N}+\left|{\epsilon\over 2M}\right||M| \\
&={\epsilon\over 2}+{\epsilon\over 2}=\epsilon.
\end{align*}
This is just what we needed, so by the official definition,
$\ds \lim_{x\to a}f(x)g(x)=LM$.
\end{proof}

A handful of such theorems give us the tools to compute many limits
without explicitly working with the definition of limit.

\begin{theorem} Suppose that $\ds \lim_{x\to a}f(x)=L$ and $\ds \lim_{x\to a}g(x)=M$ and
$k$ is some constant. Then
\begin{align*}
&\lim_{x\to a} kf(x) = k\lim_{x\to a}f(x)=kL \\
&\lim_{x\to a} (f(x)+g(x)) = \lim_{x\to a}f(x)+\lim_{x\to a}g(x)=L+M \\
&\lim_{x\to a} (f(x)-g(x)) = \lim_{x\to a}f(x)-\lim_{x\to a}g(x)=L-M \\
&\lim_{x\to a} (f(x)g(x)) = \lim_{x\to a}f(x)\cdot\lim_{x\to a}g(x)=LM \\
&\lim_{x\to a} {f(x)\over g(x)} = {\lim_{x\to a}f(x)\over\lim_{x\to
    a}g(x)}={L\over M},\hbox{ if $M$ is not 0}
\end{align*}
\label{thm:properties of limits}
\end{theorem}

Roughly speaking, these rules say that to compute the limit of an
algebraic expression, it is enough to compute the limits of the
``innermost bits'' and then combine these limits. This often means
that it is possible to simply plug in a value for the variable, since
$\ds \lim_{x\to a} x =a$.

\begin{example}
Compute $\ds
\lim_{x\to 1}{x^2-3x+5\over x-2}$. If we apply the theorem
in all its gory detail, we get
\begin{align*}
\lim_{x\to 1}{x^2-3x+5\over x-2}&=
{\lim_{x\to 1}(x^2-3x+5)\over \lim_{x\to1}(x-2)} \\
&={(\lim_{x\to 1}x^2)-(\lim_{x\to1}3x)+(\lim_{x\to1}5)\over 
  (\lim_{x\to1}x)-(\lim_{x\to1}2)} \\
&={(\lim_{x\to 1}x)^2-3(\lim_{x\to1}x)+5\over (\lim_{x\to1}x)-2} \\
&={1^2-3\cdot1+5\over 1-2} \\
&={1-3+5\over -1} = -3 \\
\end{align*}
\end{example}

It is worth commenting on the trivial limit $\ds \lim_{x\to1}5$. From one
point of view this might seem meaningless, as the number 5 can't
``approach'' any value, since it is simply a fixed number. But 5 can,
and should, be interpreted here as the function that has value 5
everywhere, $f(x)=5$, with graph a horizontal line. From this point of
view it makes sense to ask what happens to the height of the function
as $x$ approaches 1.

Of course, as we've already seen, we're primarily interested in limits
that aren't so easy, namely, limits in which a denominator approaches
zero. There are a handful of algebraic tricks that work on many of
these limits.

\begin{example}
Compute $\ds\lim_{x\to1}{x^2+2x-3\over x-1}$. We can't
simply plug in $x=1$ because that makes the denominator zero. 
However:
\begin{align*}
\lim_{x\to1}{x^2+2x-3\over x-1}&=\lim_{x\to1}{(x-1)(x+3)\over x-1} \\
&=\lim_{x\to1}(x+3)=4 \\
\end{align*}
\vskip-10pt
\end{example}

While theorem~\xrefn{thm:properties of limits} is very helpful, we
need a bit more to work easily with limits. Since the theorem applies
when some limits are already known, we need to know the behavior of
some functions that cannot themselves be constructed from the simple
arithmetic operations of the theorem, such as $\ds\sqrt{x}$. Also,
there is one other extraordinarily useful way to put functions
together: composition\index{composition of functions}. If $f(x)$ and
$g(x)$ are functions, we can form two functions by composition:
$f(g(x))$ and $g(f(x))$. For example, if $\ds f(x)=\sqrt{x}$ and $\ds
\ds g(x)=x^2+5$, then $\ds f(g(x))=\sqrt{x^2+5}$ and $\ds
g(f(x))=(\sqrt{x})^2+5=x+5$.  Here is a companion to
theorem~\xrefn{thm:properties of limits} for composition:

\begin{theorem} Suppose that $\ds \lim_{x\to a}g(x)=L$ and $\ds \lim_{x\to L}f(x)=f(L)$. Then
$$\lim_{x\to a} f(g(x)) = f(L).$$
\label{thm:limit of composition}
\end{theorem}

Note the special form of the condition on $f$: it is not enough to
know that $\ds\lim_{x\to L}f(x) = M$, though it is a bit tricky to see
why. Many of the most familiar functions do have this property, and
this theorem can therefore be applied. For example:

\begin{theorem} Suppose that $n$ is a positive integer. Then
$$\lim_{x\to a}\root n\of{x} = \root n\of{a},$$
provided that $a$ is positive if $n$ is even.
\label{thm:continuity of roots}
\end{theorem}

This theorem is not too difficult to prove from the definition of limit.

Another of the most common algebraic tricks was used in
section~\xrefn{sec:slope of a function}. Here's another example:

\begin{example}
Compute $\ds\lim_{x\to-1} {\sqrt{x+5}-2\over x+1}$.
\begin{align*}
\lim_{x\to-1} {\sqrt{x+5}-2\over x+1}&=
\lim_{x\to-1} {\sqrt{x+5}-2\over x+1}{\sqrt{x+5}+2\over \sqrt{x+5}+2} \\
&=\lim_{x\to-1} {x+5-4\over (x+1)(\sqrt{x+5}+2)} \\
&=\lim_{x\to-1} {x+1\over (x+1)(\sqrt{x+5}+2)} \\
&=\lim_{x\to-1} {1\over \sqrt{x+5}+2}={1\over4} \\
\end{align*}
At the very last step we have used theorems~\xrefn{thm:limit of
composition} and \xrefn{thm:continuity of roots}.
\end{example}

Occasionally we will need a slightly modified version of the limit
definition. Consider the function $\ds f(x)=\sqrt{1-x^2}$, the upper half of
the unit circle. What can we say about $\ds \lim_{x\to 1}f(x)$? It is
apparent from the graph of this familiar function that as $x$ gets
close to 1 from the left, the value of $f(x)$ gets close to zero. It
does not even make sense to ask what happens as $x$ approaches 1 from
the right, since $f(x)$ is not defined there. The definition of the
limit, however, demands that $f(1+\Delta x)$ be close to $f(1)$
whether $\Delta x$ is positive or negative. Sometimes the limit of a
function exists from one side or the other (or both) even though the
limit does not exist. Since it is useful to be able to talk about this
situation, we introduce the concept of 
{\dfont one sided limit\index{one sided limit}}:

\begin{definition} (One-sided limit) Suppose that $f(x)$ is a function. We say that
$\ds \lim_{x\to a^-}f(x)=L$ if for every $\epsilon>0$ there is a $\delta >
0$ so that whenever $0 < a-x < \delta$, $|f(x)-L|<\epsilon$.  We say
that $\lim_{x\to a^+}f(x)=L$ if for every $\epsilon>0$ there is a
$\delta > 0$ so that whenever $0 < x-a < \delta$, $|f(x)-L|<\epsilon$.
\end{definition}

Usually $\ds \lim_{x\to a^-}f(x)$ is read ``the limit of $f(x)$ from the
left'' and $\ds \lim_{x\to a^+}f(x)$ is read ``the limit of $f(x)$ from the
right''.

\begin{example}
Discuss $\ds\lim_{x\to 0}{x\over|x|}$, 
$\ds\lim_{x\to 0^-}{x\over|x|}$,
and $\ds\lim_{x\to 0^+}{x\over|x|}$.

The function $f(x)=x/|x|$ is undefined at 0; when $x>0$, $|x|=x$ and
so $f(x)=1$; when $x<0$, $|x|=-x$ and $f(x)=-1$. Thus
$\ds \lim_{x\to 0^-}{x\over|x|}=\lim_{x\to 0^-}-1=-1$ while 
$\ds \lim_{x\to 0^+}{x\over|x|}=\lim_{x\to 0^+}1=1$. The limit of $f(x)$
must be equal to both the left and right limits; since they are
different, the limit $\ds \lim_{x\to 0}{x\over|x|}$ does not exist.
\end{example}

\begin{exercises}

Compute the limits. If a limit does not exist, explain why.

\twocol

\begin{exercise} $\ds \lim_{x\to 3}{x^2+x-12\over x-3}$
\begin{answer} 7
\end{answer}\end{exercise}

\begin{exercise} $\ds \lim_{x\to 1}{x^2+x-12\over x-3}$
\begin{answer} 5
\end{answer}\end{exercise}

\begin{exercise} $\ds \lim_{x\to -4}{x^2+x-12\over x-3}$
\begin{answer} 0
\end{answer}\end{exercise}

\begin{exercise} $\ds \lim_{x\to 2} {x^2+x-12\over x-2}$
\begin{answer} undefined
\end{answer}\end{exercise}

\begin{exercise} $\ds \lim_{x\to 1} {\sqrt{x+8}-3\over x-1}$
\begin{answer} $1/6$
\end{answer}\end{exercise}

\begin{exercise} $\ds \lim_{x\to 0^+} \sqrt{{1\over x}+2} - \sqrt{1\over x}$.
\begin{answer} 0
\end{answer}\end{exercise}

\begin{exercise} $\ds\lim _{x\to 2} 3$
\begin{answer} 3
\end{answer}\end{exercise}

\begin{exercise} $\ds\lim _{x\to 4 } 3x^3 - 5x $
\begin{answer} 172
\end{answer}\end{exercise}

\begin{exercise} $\ds \lim _{x\to 0 } {4x - 5x^2\over x-1}$
\begin{answer} 0
\end{answer}\end{exercise}

\begin{exercise} $\ds\lim _{x\to 1 } {x^2 -1 \over x-1 }$
\begin{answer} 2
\end{answer}\end{exercise}

\begin{exercise} $\ds\lim _{x\to 0^ + } {\sqrt{2-x^2 }\over x}$
\begin{answer} does not exist
\end{answer}\end{exercise}

\begin{exercise} $\ds\lim _{x\to 0^ + } {\sqrt{2-x^2}\over x+1}$
\begin{answer} $\ds \sqrt2$
\end{answer}\end{exercise}

\begin{exercise} $\ds\lim _{x\to a } {x^3 -a^3\over x-a}$
\begin{answer} $\ds 3a^2$
\end{answer}\end{exercise}

\begin{exercise} $\ds\lim _{x\to 2 } (x^2 +4)^3$
\begin{answer} 512
\end{answer}\end{exercise}

\begin{exercise} $\ds\lim _{x\to 1 } \begin{cases}
x-5 & x\neq 1, \\
7 & x=1. \end{cases}$
\begin{answer} $-4$
\end{answer}\end{exercise}

\endtwocol

\msk
\begin{exercise} $\ds\lim _{x\to 0 } x\sin \left( {1\over x}\right)$
(Hint: Use the fact that $|\sin a |< 1 $ for any real number $a$. You
should probably use the definition of a limit here.)
\begin{answer} $0$
\end{answer}\end{exercise}

\begin{exercise} Give an $\epsilon$--$\delta$ proof, similar to
example~\xrefn{exam:epsilon-delta proof},
of the fact that 
$\ds \lim_{x\to 4} (2x-5) = 3$. 
\end{exercise}

\begin{exercise} Evaluate the expressions by reference to this graph:\hfill\break

% BADBAD
%\hbox{\epsfxsize8cm\epsfbox{limit-exercise-graph.eps}}\hfill\break

% \halign{&\indent#\hfill \\
%  (a) $\ds \lim_{x\to 4} f(x)$ & 
%  (b) $\ds \lim_{x\to -3} f(x)$ & 
%  (c) $\ds \lim_{x\to 0} f(x)$  \\
%  (d) $\ds \lim_{x\to 0^-} f(x)$ & 
%  (e) $\ds \lim_{x\to 0^+} f(x)$ & 
%  (f) $\ds f(-2)$  \\
%  (g) $\ds \lim_{x\to 2^-} f(x)$ & 
%  (h) $\ds \lim_{x\to -2^-} f(x)$ & 
%  (i) $\ds \lim_{x\to 0} f(x+1)$  \\
%  (j) $\ds f(0)$ & 
%  (k) $\ds \lim_{x\to 1^-} f(x-4)$ & 
%  (l) $\ds \lim_{x\to 0^+} f(x-2)$  \\}
% \begin{answer} (a) $8$, (b) $6$, (c) dne, (d) $-2$, (e) $-1$, (f) $8$,
%  (g) $7$, (h) $6$, (i) $3$, (j) $-3/2$, (k) $6$, (l) $2$
% \end{answer}
\end{exercise}

\begin{exercise} Use a calculator to estimate $\ds\lim_{x\to 0}
{\sin x\over x}$.
\end{exercise}

\begin{exercise} Use a calculator to estimate $\ds\lim_{x\to 0}
{\tan(3x)\over\tan(5x)}$.
\end{exercise}

\end{exercises}
