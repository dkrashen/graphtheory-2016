\section{Other bases}{}{}

Notice that if $q\in \Q$ and $a>0$ then
$a^q = e^{\ln (a^q ) } = e^{q\ln a }$.
This equation motivates the following definition.

\begin{definition}
For $a>0 $ and $x\in\R$, we define
$a^x = e^{x\ln a}$. The function $f(x)=a^x $ is the exponential function
 with base $a$.
Separately, we define $0^x =0 $ for $x>0 $.
\end{definition}

Notice also that for $x\in \R$ and $a>0$,
$\ln a^x = \ln (e^{x\ln a }) = x\ln a$.
Hence, the power rule for the natural logarithm works even when the
power is irrational.

We now show that the familiar rules for exponents are valid.

\begin{theorem} For $x,y \in \R$ and $a,b>0$:
\label{thm:general exp rules}

\begin{itemize} % BADBAD

\item{a} $a^{x+y} = a^x a^y$

\item{b} $a^{x-y} = a^x/a^y$

\item{c} $(a^x)^y = a^{xy}$

\item{d} $(ab)^x = a^x b^x$

\end{itemize}

\begin{proof}

(a) We compute:
$a^{x+y} = e^{(x+y) \ln a } = e^{x\ln a + y\ln a} =
e^{x\ln a } e^{y\ln a } = a^x a^y$.

The proof of (b) is similar and left as an exercise.

(c) We compute:
$(a^x )^y = e^{y\ln(a^x)} = e^{yx\ln a} = a^{xy}$.

(d) We compute:
$(ab)^x = e^{x\ln (ab) } = e^{x\ln a +x\ln b}
= e^{x \ln a } e^{x\ln b} = a^x b^x$.
\end{proof}


\begin{theorem} If $f(x) =a^x $ (with $a>0 $) then $f'(x) =  a^x \ln a $.
\begin{proof} 
$$f'(x) = {d\over dx}(e^{x\ln a}) = e^{x\ln a}\ln a  =a^x \ln a.$$
\vskip-20pt
\end{proof}

\cor For $a>0 $ and $a\neq 1$, $\ds\int a^x\,dx
 = {a^x\over\ln a} + C$.
\end{proof}

We are now in a position to prove the general power rule.

\begin{theorem} (Power Rule) If $f(x) = x^n $, $x>0$, and $n$ is any real number,
then $f'(x) =nx^{n-1}$.
\begin{proof}
$\ds f'(x) ={d\over dx} x^n = {d\over dx} e^{n\ln x} = 
e^{n\ln x}{n\over x}
= {nx^n \over x} = nx^{n-1}$.
\end{proof}

The restriction that $x>0 $ is necessary since we have not defined
exponential expressions with negative bases and arbitrary real powers.

We now turn to logarithms base $a$. Note that if $a>0 $ and $a\neq 1 $
 then $a^x \ln a \neq 0 $ for every $x$. Hence, the function $f(x)
 =a^x $ is injective.

\begin{definition} If $a>0$ and $a\neq1$, the inverse of $a^x$ is called the
{\dfont logarithmic function base} $a$. In symbols, we write this
function as  $\ds \log _a x$.
\end{definition}

We exclude $a=1$ because $1^x = 1$ is not injective
on any domain containing more than one point.


\begin{remark}{Remark} If $a=10 $ we usually write $\log $ instead of $\log
_{10}$, and of course $\log _e =\ln$.
In more advanced texts,
``$\log $'' refers to the natural logarithm.
\end{remark}


\begin{theorem} The following hold for $a, x, y >0 $, $a\neq 1 $, and
$q\in \R$:
\label{thm:general log rules}

\begin{itemize} % BADBAD

\item{a.} $\ds\log _a (xy) =\log _a x +\log _a y $

\item{b.} $\ds\log _a {x\over y} = \log _a x -\log _a y $

\item{c.} $\ds\log _a x^q = q\log _a x .$

\end{itemize}
\begin{proof}
(a) Let $u= \log_a x$ and $v= \log_a y$. Then $a^u =x $ and $a^v
= y$, and $\ds xy = a^u a^v = a^{u+v}$, so 
$\ds \log _a (xy) = u+v = \log_a x+\log_a y$.

 The other parts are left as exercises.
\end{proof}

 When computing decimal approximations to logs of
arbitrary bases with a calculator or a computer algebra system the
following result comes in handy.

\lem If $a, b>0 $, $a ,b \neq 1 $, and $x>0 $,
$\ds\log_a x = {\log_b  x\over \log_b a}$.
\begin{proof}
Let $y=\log_a x $, so $a^y = x$. Then
$\ds\log_b x = \log_b (a^y ) = y\log_b a =\log_a x \log _b a$.
\end{proof}

Typically this is useful when $b=e$ and $b=10$, since calculators can
typically compute logarithms to those bases.


\begin{theorem} $\ds{d\over dx} \log _a x = {1\over x\ln a}$.
\begin{proof}
By the preceding lemma, $f(x) =\ln x/\ln a$, and the derivative is
then easy.
\end{proof}

Finally, we express $e^x$ as a limit. When $x=1$ we get a limit
expression for $e$ which is sometimes taken as the definition of
$e$.

\begin{theorem} If $x\geq 0$,
$\ds e^x = \lim_{n\to\infty} \left(
1+{x\over n}\right)^n$.
\begin{proof}
If $x=0$ both expressions are 1.

If $x>0$ we begin by rewriting the right side as we have  before:
$$\left(1+{x\over n}\right)^n=\left(e^{\ln (1+x/n)}\right)^n=
e^{n\ln (1+x/n)}.$$
Now because $e^x$ is continuous,
$$\lim_{n\to\infty}e^{n\ln (1+x/n)}=e^{\lim_{n\to\infty}n\ln (1+x/n)}.$$
So really we need to compute $\ds\lim_{n\to\infty}n\ln (1+x/n)$, for
which we use L'H\^opitals rule:
$$\lim_{n\to\infty}n\ln (1+x/n)=\lim_{n\to\infty}{\ln (1+x/n)\over 1/n}=
\lim_{n\to\infty}{{1\over 1+x/n}{-x\over n^2} \over -1/n^2}=
\lim_{n\to\infty}{1\over 1+x/n}x = x.$$
\end{proof}

This same simple fact, $\ds a=e^{\ln a}$, is useful in many similar
situations.


\begin{example} Let $f(x) =x^x$, $x>0$. Compute $f'(x)$ and
$\ds\lim_{x\to0^+} f(x)$.

Start with $f(x)=x^x = e^{x\ln x}$. Then 
$$f'(x)=e^{x\ln x}\left(x{1\over x}+\ln x\right)=x^x(1+\ln x).$$
For the limit, we again notice that
$$\lim_{x\to0^+} x^x = \lim_{x\to0^+}e^{x\ln x}
=e^{\lim_{x\to0^+} x\ln x}.$$
Then we compute the limit by L'H\^opital's rule again:
$$\lim_{x\to0^+}x\ln x = \lim_{x\to0^+}{\ln x\over 1/x}=
\lim_{x\to0^+}{1/x\over -1/x^2}=\lim_{x\to0^+}(-x)=0.$$
Thus $\ds\lim_{x\to0^+} x^x=e^0=1$.
\end{example}

\begin{example} Compute $\ds\int_{\pi/6}^{\pi/3} 2^{\cos x} \sin x\; dx$.

Let $u=\cos x$, so $du=-\sin x\;dx $. Changing the limits, when
$x=\pi/6$, $u =\sqrt{3}/2$, and when $x=\pi/3$,
$u=1/2$. Then
$$\int_{\pi/6}^{\pi/3} 2^{\cos x} \sin x\;dx
 =-\int_{\sqrt{3}/2}^{1/2} 2^u du
= \left.-{2^u\over\ln 2}\right|_{\sqrt{3}/2}^{1/2} 
  ={-2^{1/2} +2^{\sqrt{3}/2}\over\ln 2}.$$
\end{example}

\begin{exercises}

\begin{exercise} Prove part (b) of theorem~\xrefn{thm:general exp rules}.

\begin{exercise} Sketch the graph of $\ds y=a^x $ in the three cases $a>1$,
$a=1$, and $0<a< 1 $. What happens to the graph as $a\to 0^+$?
What happens to the graph as $a\to\infty $?

\begin{exercise} Sketch the graph of $\ds y=\log_a x$ in the two cases $a>1$ and
$0<a< 1 $. What happens to the graph as $a\to 0^+$?  What happens to
the graph as $a\to\infty$? (Use the previous exercise together with
exercise~\xrefn{sec:inverse
  functions}.\xrefn{exer:inverse is reflection}.)


\begin{exercise} Prove parts (b) and (c) of theorem~\xrefn{thm:general log rules}.

\begin{exercise} Sketch the graph of $\ds y=3^{6x-1} + 5 $.

\begin{exercise} Sketch the graph of $\ds y= -(1/2)^{-3x}$. 

\begin{exercise} Sketch the graph of $\ds y=4\log_2 (12x + 6)-2$.

\begin{exercise} Compute the second derivative of $\ds f(x)=x^x$.

\begin{exercise} Compute $\ds f'(\pi/4)$ when $\ds f(x)
= 5^{\sin 3x} + \log_7 x$. 

\begin{exercise} Compute the derivative of 
$\ds f(x) = 3^x - 4x^2 + \sin(3x) - \pi^e$. 

\begin{exercise} Compute $\ds \int_1^2 3^x - x^3\,dx$.

\begin{exercise} Compute $\int \sin (2^x ) 2^x dx $.


\begin{exercise} Find the area of the region given by
$\ds \{(x, y) \mid 1 \leq x \leq 2 , 2^x \leq y \leq 3^x\}$.

\begin{exercise} Find the average of the function $\ds f(x)= 2x\cdot 5^{x^2}$
on the interval $[4,9]$.


\begin{exercise} Find the volume of the solid obtained by rotating the
region
$\ds \{(x,y) \mid 2 \leq x \leq 4, (\log _2 x)/x
\leq y \leq 2^x \}$ about the line $y=-1 $. 

 \begin{exercise} Show that $\log_a x =-\log_{1/a} x $ for any
 $a>0, a\neq 1$.
 Interpret this result geometrically; that is, sketch the graph of
 $y=\log_a x$ and $y=\log_{1/a} x$ on the same diagram and
 point out how the graphs are related to each other.

\end{exercises}

