\section{A hard limit}{}{}
\nobreak
We want to compute this limit:
$$\lim_{\Delta x\to0} {\sin\Delta x\over \Delta x}.$$
Equivalently, to make the notation a bit simpler, we can compute
$$\lim_{x\to0} {\sin x\over x}.$$
In the original context we need to keep $x$ and $\Delta x$ separate,
but here it doesn't hurt to rename $\Delta x$ to something more
convenient. 

To do this we
need to be quite clever, and to employ some indirect reasoning. The
indirect reasoning is embodied in a theorem, frequently called the
{\dfont squeeze theorem\index{squeeze theorem}.}

\begin{theorem} (Squeeze Theorem)
Suppose that $g(x) \le f(x) \le h(x)$ for all $x$ close to $a$ but not
equal to $a$. If $\lim_{x\to a}g(x)=L=\lim_{x
\to a}h(x)$, then $\lim_{x\to a}f(x)=L$.
\label{thm:squeeze theorem}
\end{theorem}

This theorem can be proved using the official definition of limit. We
won't prove it here, but point out that it is easy to understand and
believe graphically. The condition says that $f(x)$ is trapped between
$g(x)$ below and $h(x)$ above, and that at $x=a$, both $g$ and $h$
approach the same value. This means the situation looks something like
figure~\xrefn{fig:squeeze}. The wiggly curve is $x^2\sin(\pi/x)$, the
upper and lower curves are $x^2$ and $-x^2$. Since the sine function
is always between $-1$ and $1$, $-x^2\le x^2\sin(\pi/x)\le x^2$, and
it is easy to see that $\lim_{x\to0}-x^2=0=\lim_{x\to0}x^2$.
It is not so easy to see directly, that is algebraically, that 
$\lim_{x\to0}x^2\sin(\pi/x)=0$, because the $\pi/x$ prevents us from
simply plugging in $x=0$. The squeeze theorem makes this ``hard
limit'' as easy as the trivial limits involving $x^2$.

% BADBAD
% \figure
% \vbox{\beginpicture
% \normalgraphs
% \ninepoint
% \setcoordinatesystem units <4truecm,4truecm>
% \setplotarea x from -1 to 1, y from -1 to 1
% \axis left shiftedto x=0 /
% \axis bottom shiftedto y=0 /
% \setquadratic
% \plot -1.000 0.000 -0.980 0.060 -0.961 0.118 -0.941 0.173 -0.922 0.224
% -0.902 0.272 -0.882 0.317 -0.863 0.357 -0.843 0.392 -0.824 0.423
% -0.804 0.448 -0.784 0.468 -0.765 0.481 -0.745 0.488 -0.726 0.488
% -0.706 0.481 -0.686 0.467 -0.667 0.445 -0.647 0.415 -0.628 0.377
% -0.608 0.332 -0.588 0.280 -0.569 0.223 -0.549 0.161 -0.530 0.096
% -0.510 0.032 -0.490 -0.030 -0.471 -0.084 -0.451 -0.128 -0.432 -0.156
% -0.412 -0.165 -0.392 -0.152 -0.373 -0.117 -0.353 -0.063 -0.334 -0.001
% -0.314 0.054 -0.294 0.082 -0.275 0.068 -0.255 0.016 -0.236 -0.039
% -0.216 -0.043 -0.196 0.011 -0.177 0.028 -0.157 -0.022 -0.138 0.014
% -0.118 -0.014 -0.098 -0.005 -0.079 -0.005 -0.059 -0.001 -0.040 0.001
% -0.020 0.000 0 0
% 0.020 0.000 0.040 -0.001 0.059 0.001 0.079 0.005 0.098 0.005
% 0.118 0.014 0.138 -0.014 0.157 0.022 0.177 -0.028 0.196 -0.011
% 0.216 0.043 0.236 0.039 0.255 -0.016 0.275 -0.068 0.294 -0.082
% 0.314 -0.054 0.334 0.001 0.353 0.063 0.373 0.117 0.392 0.152
% 0.412 0.165 0.432 0.156 0.451 0.128 0.471 0.084 0.490 0.030
% 0.510 -0.032 0.530 -0.096 0.549 -0.161 0.569 -0.223 0.588 -0.280
% 0.608 -0.332 0.628 -0.377 0.647 -0.415 0.667 -0.445 0.686 -0.467
% 0.706 -0.481 0.726 -0.488 0.745 -0.488 0.765 -0.481 0.784 -0.468
% 0.804 -0.448 0.824 -0.423 0.843 -0.392 0.863 -0.357 0.882 -0.317
% 0.902 -0.272 0.922 -0.224 0.941 -0.173 0.961 -0.118 0.980 -0.060
% 1.000 0.000 /
% \plot -1 1 0 0 1 1 /
% \plot -1 -1 0 0 1 -1 /
% \endpicture}
% \figrdef{fig:squeeze}
% \endfigure{The squeeze theorem.}

To do the hard limit that we want, $\lim_{x\to0}
(\sin x)/x$, we will find two simpler functions $g$ and $h$ so that 
$g(x)\le (\sin x)/x\le h(x)$, and so that
$\lim_{x\to0}g(x)=\lim_{x\to0}h(x)$. Not too surprisingly, this will
require some trigonometry and geometry. Referring to
figure~\xrefn{fig:hard limit}, $x$ is the measure of the angle in
radians. Since the circle has radius 1, the coordinates of point $A$
are $(\cos x,\sin x)$, and the area of the small triangle is 
$(\cos x\sin x)/2$. This triangle is completely contained within the
circular wedge-shaped region bordered by two lines and the circle from
$(1,0)$ to point $A$. Comparing the areas of the triangle and the
wedge we see
$(\cos x\sin x)/2 \le x/2$, since the area of a circular region with
angle $\theta$ and radius $r$ is $\theta r^2/2$. With a little algebra
this turns into $(\sin x)/x \le 1/\cos x$, giving us the $h$ we seek.

% BADBAD
% \figure
% \vbox{\beginpicture
% \normalgraphs
% \ninepoint
% \setcoordinatesystem units <4truecm,4truecm>
% \setplotarea x from 0 to 1, y from 0 to 1
% \circulararc 90 degrees from 1 0 center at 0 0
% \axis left ticks numbered from 0 to 1 by 1 /
% \axis bottom ticks numbered from 1 to 1 by 1 /
% \putrule from  0.8660254040 0 to 0.8660254040 0.5
% \putrule from  1 0 to 1 .577
% \plot 0 0 1 .577 /
% \put {$x$} [bl] <5pt,2.5pt> at 0.1 0
% \put {$A$} [r] <-6pt,2pt> at 0.8660254040 0.5
% \put {$B$} [bl] <3pt,3pt> at 1  .577
% \circulararc 30 degrees from 0.1 0 center at 0 0
% \endpicture}
% \figrdef{fig:hard limit}
% \endfigure{Visualizing $\sin x / x$.}

To find $g$, we note that the circular wedge is completely contained
inside the larger triangle. The height of the triangle, from $(1,0)$
to point $B$, is $\tan x$, so comparing areas we get
$x/2 \le (\tan x)/2 = \sin x / (2\cos x)$. With a little algebra this
becomes $\cos x \le (\sin x)/x$. So now we have 
$$ \cos x \le {\sin x\over x}\le {1\over\cos x}.$$
Finally, the two limits $\lim_{x\to0}\cos x$ and $\lim_{x\to0}1/\cos x$
are easy, because $\cos(0)=1$. By the squeeze theorem,
$\lim_{x\to0} (\sin x)/x = 1$ as well.

Before we can complete the calculation of the derivative of the sine,
we need one other limit:
$$\lim_{x\to0}{\cos x - 1\over x}.$$
This limit is just as hard as $\sin x/x$, but closely related to it,
so that we don't have to a similar calculation; instead we can do a
bit of tricky algebra.
$${\cos x - 1\over x}={\cos x - 1\over x}{\cos x+1\over\cos x+1}
={\cos^2 x - 1\over x(\cos x+1)}={-\sin^2 x\over x(\cos x+1)}=
-{\sin x\over x}{\sin x\over \cos x + 1}.$$
To compute the desired limit it is sufficient to compute the limits of
the two final fractions, as $x$ goes to 0. The first of these is the
hard limit we've just done, namely 1. The second turns out to be
simple, because the denominator presents no problem:
$$\lim_{x\to0}{\sin x\over \cos x + 1}={\sin 0\over \cos 0+1}=
{0\over 2}  = 0.$$
Thus,
$$\lim_{x\to0}{\cos x - 1\over x}=0.$$

\begin{exercises}

\twocol

\begin{exercise} Compute $\ds\lim_{x\to 0} {\sin (5x)\over x}$
\begin{answer} $5$
\end{answer}\end{exercise}

%(Hint: make the substitution $y=5x $ and note
% that $y\to 0 $ as $x\to 0 $.)


\begin{exercise} Compute $\ds\lim_{x\to 0 } {\sin(7x)\over\sin (2x)}$
\begin{answer} $7/2$
\end{answer}\end{exercise}

\begin{exercise}  Compute $\ds\lim_{x\to 0 } {\cot (4x) \over\csc (3x)}$
\begin{answer} $3/4$
\end{answer}\end{exercise}

\begin{exercise} Compute $\ds\lim_{x\to 0 } {\tan x\over x}$
\begin{answer} $1$
\end{answer}\end{exercise}

\endtwocol

\begin{exercise} Compute $\ds\lim_{x\to \pi/4} {\sin x
    -\cos x \over\cos (2x)}$
 \begin{answer} $-\sqrt2/2$
\end{answer}\end{exercise}

\begin{exercise}  For all $x\geq 0$, $4x-9 \leq f(x) \leq x^2 - 4x +7$. Find
  $\ds\lim_{x\to4}f(x)$.
\begin{answer} $7$
\end{answer}\end{exercise}

\begin{exercise}  For all $x$, $2x \leq g(x) \leq x^4 - x^2 +2$. Find
  $\ds\lim_{x\to1}g(x)$.
\begin{answer} $2$
\end{answer}\end{exercise}

\begin{exercise} Use the Squeeze Theorem to show that $\ds\lim_{x\to0} x^4
 \cos(2/x)=0$.
\end{exercise}

\end{exercises}
