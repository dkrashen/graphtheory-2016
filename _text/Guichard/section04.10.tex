\section{Limits revisited}{}{}
\nobreak
We have defined and used the concept of limit, primarily in our
development of the derivative. Recall that $\ds \lim_{x\to a}f(x)=L$ is
true if, in a precise sense, $f(x)$ gets closer and closer to $L$ as
$x$ gets closer and closer to $a$. While some limits are easy to see,
others take some ingenuity; in particular, the limits that define
derivatives are always difficult on their face, since in 
$$\lim_{\Delta x\to 0} {f(x+\Delta x)-f(x)\over \Delta x}$$
both the numerator and denominator approach zero. Typically this
difficulty can be resolved when $f$ is a ``nice'' function and we are
trying to compute a derivative. Occasionally such limits are
interesting for other reasons, and the limit of a fraction in which
both numerator and denominator approach zero can be difficult to
analyze. Now that we have the derivative available, there is another
technique that can sometimes be helpful in such circumstances.

Before we introduce the technique, we will also expand our concept of
limit. We will occasionally want to know what happens to some quantity
when a variable gets very large or ``goes to infinity''.

\begin{example}
What happens to the function $\ds \cos(1/x)$ as $x$ goes to infinity? It
seems clear that as $x$ gets larger and larger, $1/x$ gets closer and
closer to zero, so $\cos(1/x)$ should be getting closer and closer to 
$\cos(0)=1$.
\end{example}

As with ordinary limits, this concept of ``limit at infinity'' can be
made precise. Roughly, we want $\ds \lim_{x\to \infty}f(x)=L$ to mean that
we can make $f(x)$ as close as we want to $L$ by making $x$ large
enough. Compare this definition to the definition of limit in 
section~\xrefn{sec:limits}.

\begin{definition} (Limit at infinity) If $f$ is a function, we say that $\ds \lim_{x\to
  \infty}f(x)=L$ if for every $\epsilon>0$ there is an $N > 0$ so that
  whenever $x>N$, $|f(x)-L|<\epsilon$.
We may similarly define $\ds \lim_{x\to-\infty}f(x)=L$.
\index{limit at infinity}\label{defn:limit at infinity}
\end{definition}

We include this definition for completeness, but we will not explore it in
detail. Suffice it to say that such limits behave in much the same way
that ordinary limits do; in particular there is a direct analog of 
theorem~\xrefn{thm:properties of limits}.

Now consider this limit:
$$\lim_{x\to \pi}{x^2-\pi^2\over \sin x}.$$
As $x$ approaches $\pi$, both the numerator and denominator approach
zero, so it is not obvious what, if anything, the quotient
approaches. We can often compute such limits by application of the
following theorem.

\begin{theorem} (L'H\^opital's Rule) For ``sufficiently nice'' functions $f(x)$
and $g(x)$, if $\ds\lim_{x\to a} f(x)= 0 = \lim_{x\to a}
g(x)$ or $\ds\lim_{x\to a} f(x)= \pm\infty = \lim_{x\to a}
g(x)$, and if $\ds\lim_{x\to a}{f'(x)\over g'(x)}$ exists,
then $\ds\lim_{x\to a}{f(x)\over g(x)}=\lim_{x\to
a}{f'(x)\over g'(x)}$.  This remains true if ``$x\to a$'' is replaced
by ``$x\to \infty$'' or ``$x\to -\infty$''.
\end{theorem}
\index{L'H\^opital's Rule}

This theorem is somewhat difficult to prove, in part because it
incorporates so many different possibilities, so we will not prove it
here. We also will not need to worry about the precise definition of 
``sufficiently nice'', as the functions we encounter will be
suitable. 

%% fixme: do plausibility argument with tan lines at x=a?

\begin{example}
Compute $\ds\lim_{x\to \pi}{x^2-\pi^2\over \sin x}$ in two ways.
\msk
First we use L'H\^opital's Rule: Since the numerator and denominator
both approach zero,
$$\lim_{x\to \pi}{x^2-\pi^2\over \sin x}=
\lim_{x\to \pi}{2x \over \cos x},$$
provided the latter exists. But in fact this is an easy limit, since
the denominator now approaches $-1$, so 
$$\lim_{x\to \pi}{x^2-\pi^2\over \sin x}={2\pi\over -1} = -2\pi.$$

We don't really need L'H\^opital's Rule to do this limit. Rewrite it
as 
$$\lim_{x\to \pi}(x+\pi){x-\pi\over \sin x}$$
and note that 
$$\lim_{x\to \pi}{x-\pi\over \sin x}=
\lim_{x\to \pi}{x-\pi\over -\sin (x-\pi)}=
\lim_{x\to 0}-{x\over \sin x}$$
since $x-\pi$ approaches zero as $x$ approaches $\pi$.
Now
$$\lim_{x\to \pi}(x+\pi){x-\pi\over \sin x}=
\lim_{x\to \pi}(x+\pi)\lim_{x\to 0}-{x\over \sin x}=
2\pi(-1)=-2\pi$$
as before.
\end{example}

\begin{example} Compute $\ds\lim_{x\to \infty}{2x^2-3x+7\over
x^2+47x+1}$ in two ways.
\msk
As $x$ goes to infinity both the numerator and denominator go to
infinity, so we may apply L'H\^opital's Rule:
$$\lim_{x\to \infty}{2x^2-3x+7\over x^2+47x+1}=
\lim_{x\to \infty}{4x-3\over 2x+47}.$$
In the second quotient, it is still the case that the numerator and
denominator both go to infinity, so we are allowed to use
L'H\^opital's Rule again:
$$\lim_{x\to \infty}{4x-3\over 2x+47}=\lim_{x\to \infty}{4\over 2}=2.$$
So the original limit is 2 as well.

Again, we don't really need L'H\^opital's Rule, and in fact a more
elementary approach is easier---we divide the numerator and
denominator by $\ds x^2$:
$$\lim_{x\to \infty}{2x^2-3x+7\over x^2+47x+1}=
\lim_{x\to \infty}{2x^2-3x+7\over x^2+47x+1}{{1\over x^2}\over {1\over
    x^2}}=
\lim_{x\to \infty}{2-{3\over x}+{7\over x^2}\over
1+{47\over x}+{1\over x^2}}.$$
Now as $x$ approaches infinity, all the quotients with some power of
$x$ in the denominator approach zero, leaving 2 in the numerator and 1
in the denominator, so the limit again is 2.
\end{example}

\begin{example} Compute $\ds\lim_{x\to 0}{\sec x - 1\over \sin x}$.
\msk
Both the numerator and denominator approach zero, so applying 
L'H\^opital's Rule:
$$\lim_{x\to 0}{\sec x - 1\over \sin x}=
\lim_{x\to 0}{\sec x\tan x\over \cos x}={1\cdot 0\over 1}=0.$$
\vskip -16pt
\end{example}

\begin{example} Compute $\ds\lim_{x\to 0^+} x\ln x$.

This doesn't appear to be suitable for L'H\^opital's Rule, but it also
is not ``obvious''. As $x$ approaches zero, $\ln x$ goes to $-\infty$,
so the product looks like $(\hbox{something very small})\cdot 
(\hbox{something very large and negative})$. But this could be
anything: it depends on {\it how small\/} and {\it how large}. 
For example, consider $\ds (x^2)(1/x)$, $(x)(1/x)$, and $\ds (x)(1/x^2)$. As
$x$ approaches zero, each of these is $(\hbox{something very small})\cdot 
(\hbox{something very large})$, yet the limits are respectively 
zero, $1$, and $\infty$.

We can
in fact turn this into a L'H\^opital's Rule problem:
$$x\ln x = {\ln x\over 1/x}={\ln x\over x^{-1}}.$$
Now as $x$ approaches zero, both the numerator and denominator
approach infinity (one $-\infty$ and one $+\infty$, but only the size
is important). Using  L'H\^opital's Rule:
$$\lim_{x\to 0^+} {\ln x\over x^{-1}}=
\lim_{x\to 0^+} {1/x\over -x^{-2}} =\lim_{x\to 0^+} {1\over x}(-x^2)=
\lim_{x\to 0^+} -x = 0.$$
One way to interpret this is that since $\ds\lim_{x\to
  0^+}x\ln x = 0$, the $x$ approaches zero much faster than the $\ln x$
approaches $-\infty$.
\end{example}

% Most from Keisler
\begin{exercises}

Compute the limits.

\twocol

\begin{exercise} $\ds\lim_{x\to 0} {\cos x -1\over \sin x}$
\begin{answer} $0$
\end{answer}\end{exercise}

\begin{exercise} $\ds\lim_{x\to \infty} {e^x\over x^3}$
\begin{answer} $\infty$
\end{answer}\end{exercise}

\begin{exercise} $\ds\lim_{x\to \infty} \sqrt{x^2+x}-\sqrt{x^2-x}$
\begin{answer} $1$
\end{answer}\end{exercise}

\begin{exercise} $\ds\lim_{x\to \infty} {\ln x\over x}$
\begin{answer} $0$
\end{answer}\end{exercise}

\begin{exercise} $\ds\lim_{x\to \infty} {\ln x\over \sqrt{x}}$
\begin{answer} $0$
\end{answer}\end{exercise}

\begin{exercise} $\ds\lim_{x\to\infty} {e^x + e^{-x}\over e^x -e^{-x}}$
\begin{answer} 1
\end{answer}\end{exercise}

\begin{exercise} $\ds\lim_{x\to0}{\sqrt{9+x}-3\over x}$
\begin{answer} $1/6$
\end{answer}\end{exercise}

\begin{exercise} $\ds\lim_{t\to1^+}{(1/t)-1\over t^2-2t+1}$
\begin{answer} $-\infty$
\end{answer}\end{exercise}

\begin{exercise} $\ds\lim_{x\to2}{2-\sqrt{x+2}\over 4-x^2}$
\begin{answer} $1/16$
\end{answer}\end{exercise}

\begin{exercise} $\ds\lim_{t\to\infty}{t+5-2/t-1/t^3\over 3t+12-1/t^2}$
\begin{answer} $1/3$
\end{answer}\end{exercise}

\begin{exercise} $\ds\lim_{y\to\infty}{\sqrt{y+1}+\sqrt{y-1}\over y}$
\begin{answer} $0$
\end{answer}\end{exercise}

\begin{exercise} $\ds\lim_{x\to1}{\sqrt{x}-1\over \root 1/3\of{x}-1}$
\begin{answer} $3/2$
\end{answer}\end{exercise}

\begin{exercise} $\ds\lim_{x\to0}{(1-x)^{1/4}-1\over x}$
\begin{answer} $-1/4$
\end{answer}\end{exercise}

\begin{exercise} $\ds\lim_{t\to 0}{\left(t+{1\over t}\right)((4-t)^{3/2}-8)}$
\begin{answer} $-3$
\end{answer}\end{exercise}

\begin{exercise} $\ds\lim_{t\to 0^+}\left({1\over t}+{1\over\sqrt{t}}\right)
(\sqrt{t+1}-1)$
\begin{answer} $1/2$
\end{answer}\end{exercise}

\begin{exercise} $\ds\lim_{x\to 0}{x^2\over\sqrt{2x+1}-1}$
\begin{answer} $0$
\end{answer}\end{exercise}

\begin{exercise} $\ds\lim_{u\to 1}{(u-1)^3\over (1/u)-u^2+3/u-3}$
\begin{answer} $-1$
\end{answer}\end{exercise}

\begin{exercise} $\ds\lim_{x\to 0}{2+(1/x)\over 3-(2/x)}$
\begin{answer} $-1/2$
\end{answer}\end{exercise}

\begin{exercise} $\ds\lim_{x\to 0^+}{1+5/\sqrt{x}\over 2+1/\sqrt{x}}$
\begin{answer} $5$
\end{answer}\end{exercise}

\begin{exercise} $\ds\lim_{x\to 0^+}{3+x^{-1/2}+x^{-1}\over 2+4x^{-1/2}}$
\begin{answer} $\infty$
\end{answer}\end{exercise}

\begin{exercise} $\ds\lim_{x\to\infty}{x+x^{1/2}+x^{1/3}\over x^{2/3}+x^{1/4}}$
\begin{answer} $\infty$
\end{answer}\end{exercise}

\begin{exercise} $\ds\lim_{t\to\infty}
{1-\sqrt{t\over t+1}\over 2-\sqrt{4t+1\over t+2}}$
\begin{answer} $2/7$
\end{answer}\end{exercise}

\begin{exercise} $\ds\lim_{t\to\infty}{1-{t\over t-1}\over 1-\sqrt{t\over t-1}}$
\begin{answer} $2$
\end{answer}\end{exercise}

\begin{exercise} $\ds\lim_{x\to-\infty}{x+x^{-1}\over 1+\sqrt{1-x}}$
\begin{answer} $-\infty$
\end{answer}\end{exercise}

\begin{exercise} $\ds\lim_{x\to1}{\int_1^x 1/t\,dt\over\int_1^x 1/(2t+1)\,dt}$
\begin{answer} $3$
\end{answer}\end{exercise}

\begin{exercise} $\ds\lim_{x\to\infty}{\int_1^x\sqrt{t+(1/t)}\,dt\over x\sqrt{x}}$
\begin{answer} $2/3$
\end{answer}\end{exercise}

\begin{exercise} $\ds\lim_{x\to\pi/2}{\cos x\over (\pi/2)-x}$
\begin{answer} $1$
\end{answer}\end{exercise}

\begin{exercise} $\ds\lim_{x\to0}{e^x-1\over x}$
\begin{answer} $1$
\end{answer}\end{exercise}

\begin{exercise} $\ds\lim_{x\to0}{x^2\over e^x-x-1}$
\begin{answer} $2$
\end{answer}\end{exercise}

\begin{exercise} $\ds\lim_{x\to1}{\ln x\over x-1}$
\begin{answer} $1$
\end{answer}\end{exercise}

\begin{exercise} $\ds\lim_{x\to0}{\ln(x^2+1)\over x}$
\begin{answer} $0$
\end{answer}\end{exercise}

\begin{exercise} $\ds\lim_{x\to1}{x\ln x\over x^2-1}$
\begin{answer} $1/2$
\end{answer}\end{exercise}

\begin{exercise} $\ds\lim_{x\to0}{\sin(2x)\over\ln(x+1)}$
\begin{answer} $2$
\end{answer}\end{exercise}

\begin{exercise} $\ds\lim_{x\to1}{x^{1/4}-1\over x}$
\begin{answer} $0$
\end{answer}\end{exercise}

\begin{exercise} $\ds\lim_{x\to1^+}{\sqrt{x}\over x-1}$
\begin{answer} $\infty$
\end{answer}\end{exercise}

\begin{exercise} $\ds\lim_{x\to1}{\sqrt{x}-1\over x-1}$
\begin{answer} $1/2$
\end{answer}\end{exercise}

\begin{exercise} $\ds\lim_{x\to\infty}{x^{-1}+x^{-1/2}\over x+x^{-1/2}}$
\begin{answer} $0$
\end{answer}\end{exercise}

\begin{exercise} $\ds\lim_{x\to\infty}{x+x^{-2}\over 2x+x^{-2}}$
\begin{answer} $1/2$
\end{answer}\end{exercise}

\begin{exercise} $\ds\lim_{x\to\infty}{5+x^{-1}\over 1+2x^{-1}}$
\begin{answer} $5$
\end{answer}\end{exercise}

\begin{exercise} $\ds\lim_{x\to\infty}{4x\over\sqrt{2x^2+1}}$
\begin{answer} $\ds 2\sqrt2$
\end{answer}\end{exercise}

\begin{exercise} $\ds\lim_{x\to0}{3x^2+x+2\over x-4}$
\begin{answer} $-1/2$
\end{answer}\end{exercise}

\begin{exercise} $\ds\lim_{x\to0}{\sqrt{x+1}-1\over \sqrt{x+4}-2}$
\begin{answer} $2$
\end{answer}\end{exercise}

\begin{exercise} $\ds\lim_{x\to0}{\sqrt{x+1}-1\over \sqrt{x+2}-2}$
\begin{answer} $0$
\end{answer}\end{exercise}

\begin{exercise} $\ds\lim_{x\to0^+}{\sqrt{x+1}+1\over\sqrt{x+1}-1}$
\begin{answer} $\infty$
\end{answer}\end{exercise}

\begin{exercise} $\ds\lim_{x\to0}{\sqrt{x^2+1}-1\over\sqrt{x+1}-1}$
\begin{answer} $0$
\end{answer}\end{exercise}

\begin{exercise} $\ds\lim_{x\to\infty}{(x+5)\left({1\over 2x}+{1\over x+2}\right)}$
\begin{answer} $3/2$
\end{answer}\end{exercise}

\begin{exercise} $\ds\lim_{x\to0^+}{(x+5)\left({1\over 2x}+{1\over x+2}\right)}$
\begin{answer} $\infty$
\end{answer}\end{exercise}

\begin{exercise} $\ds\lim_{x\to1}{(x+5)\left({1\over 2x}+{1\over x+2}\right)}$
\begin{answer} $5$
\end{answer}\end{exercise}

\begin{exercise} $\ds\lim_{x\to2}{x^3-6x-2\over x^3+4}$
\begin{answer} $-1/2$
\end{answer}\end{exercise}

\begin{exercise} $\ds\lim_{x\to2}{x^3-6x-2\over x^3-4x}$
\begin{answer} does not exist
\end{answer}\end{exercise}

\begin{exercise} $\ds\lim_{x\to1+}{x^3+4x+8\over 2x^3-2}$
\begin{answer} $\infty$
\end{answer}\end{exercise}

\endtwocol

\msk
\begin{exercise} The function $\ds f(x) = {x\over\sqrt{x^2+1}}$ has two horizontal
 asymptotes.  Find them and give a rough sketch of $f$ with its horizontal
 asymptotes. 
\begin{answer} $y=1$ and $y=-1$
\end{answer}\end{exercise}

\end{exercises}

